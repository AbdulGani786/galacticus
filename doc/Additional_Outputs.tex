\chapter{Additional Output Quantities}

\section{Halo Model Quantities}\label{sec:HaloModelOutput}\index{halo model}\index{clustering!halo model}

The following quantities related to galaxy clustering are output if {\tt [outputHaloModelData]} is set to true:
\begin{description}
 \item [{\tt nodeBias}] The large scale, lineary theory bias for each node. For satellite nodes, this corresponds to the bias of their host halo;
 \item [{\tt isolatedHostIndex}] The index of the isolated node in which this node lives. This is identical to {\tt nodeIndex} for non-satellite nodes.
\end{description}
In addition to these quantities output for each node, setting {\tt [outputHaloModelData]}$=${\tt true} causes the creation of a {\tt haloModel} group in the \glc\ output file. This group contains the following:
\begin{description}
 \item [{\tt wavenumber}] A dataset giving the wavenumbers (in units of Mpc$^{-1}$) at which all output power spectra are tabulated. The minimum and maximum wavenumbers to tabulate are determined by the {\tt [haloModelWavenumberMinimum]} and {\tt [haloModelWavenumberMaximum]} parameters respectively, while the number of points to tabulate in each decade of wavenumber is determined by the {\tt [haloModelWavenumberPointsPerDecade]} parameter.
 \item [{\tt powerSpectrum}] A dataset giving the linear theory power spectrum (in units of Mpc$^3$ normalized to $z=0$ at each wavenumber specified in the {\tt wavenumber} dataset.
 \item [{\tt Output\{i\}/mergerTree\{j\}/fourierProfile\{k\}}] A dataset giving the Fourier transform of the dark matter halo density profile (dimensionless and normalized to unity at small wavenumber) for the node with index {\tt k} in merger tree with index {\tt j} at output number {\tt i}. Profiles are written only for nodes which are isolated, and are tabulated at the wavenumbers given in the {\tt wavenumber} group. Note that wavenumbers are assumed to be comoving.
\end{description}
Finally, each numbered output group is given two additional attributes, {\tt linearGrowthFactor} and {\tt linearGrowthFactorLogDerivative} which give the growth factor, $D$, and its logarithmic derivative, $\d \ln D / \d \ln a$ at the output time.

The information output can be used to construct galaxy power spectra and correlation functions (see \S\ref{sec:ClusteringHaloModel} for example).

\section{Merger Tree Links}\index{merger tree!links}

The following properties are output to permit the merger tree structure to be recovered:
\begin{description}
 \item [{\tt nodeIndex}] A unique (within a tree) integer index identifying the node;
 \item [{\tt childIndex}] The index of this node's primary child node (or $-1$ if it has no child);
 \item [{\tt parentIndex}] The index of this node's parent node (or $-1$ if it has no parent);
 \item [{\tt siblingIndex}] The index of this node's sibling node (or $-1$ if it has no sibling);
 \item [{\tt satelliteIndex}] The index of this node's first satellite node (or $-1$ if it has no satellites);
\end{description}

\section{Virial Quantities}\index{virial}

The following quantities related to the virialized region of each node are output if {\tt outputVirialData} is set to true:
\begin{description}
 \item [{\tt nodeVirialRadius}] The virial radius (following whatever definition of virial overdensity was selected in \glc) in units of Mpc;
 \item [{\tt nodeVirialVelocity}] The circular velocity at the virial radius (in km/s).
\end{description}

\section{Merger Tree Structure}\index{merger tree!structure}

The structure of each merger tree can optionally be dumped to a file suitable for post-processing with \href{http://www.graphviz.org/}{\sc dot} after every step of evolution. To request this output set {\tt [mergerTreesDumpStructure]}$=${\tt true}. After each evolution step the tree structure will be dumped to a file named {\tt mergerTreeDump:$\langle$treeIndex$\rangle$:$\langle$outputIndex$\rangle$.gv} where $\langle${\tt treeIndex}$\rangle$ is the index of the tree and $\langle${\tt outputIndex}$\rangle$ is an incremental counter that tracks the number of outputs for this tree. These files can be processed with \href{http://www.graphviz.org/}{\sc dot} to produce a diagram of the tree structure. The node currently being evolved will be highlighted in green. This output option makes use of the \hyperlink{objects.merger_trees.dump.F90:merger_trees_dump}{\tt Merger\_Trees\_Dump} module to create the outputs.