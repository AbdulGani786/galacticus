\chapter{Auxilliary Methods}

\section{Conditional Stellar Mass Function}

Empirical conditional stellar mass functions are used by \glc\ in calculations of halo mass function sampling. \glc\ implements the following calculations of tree processing times, which can be selected via the {\normalfont \ttfamily [conditionalStellarMassFunctionMethod]} input parameter.

\subsection{Behroozi (2010) Method}

Currently the only option, and selected using {\normalfont \ttfamily [conditionalStellarMassFunctionMethod]}$=${\normalfont \ttfamily Behroozi2010}, this method adopts the fitting function of \cite{behroozi_comprehensive_2010}:
\begin{equation}
 \langle N_{\mathrm c}(M_\star|M)\rangle \equiv \int_{M_\star}^\infty \phi_{\mathrm c}(M_\star^\prime) \d \ln M_\star^\prime = {1 \over 2} \left[ 1 - \hbox{erf}\left( {\log_{10}M_\star - \log_{10} f_{\mathrm SHMR}(M) \over \sqrt{2}\sigma_{\log M_\star}} \right) \right].
\end{equation}
Here, the function $f_{\mathrm SHMR}(M)$ is the solution of
\begin{equation}
 \log_{10}M = \log_{10}M_1 + \beta \log_{10}\left({M_\star \over M_{\star,0}}\right) + {(M_\star/M_{\star,0})^\delta \over 1 + (M_\star/M_{\star,0})^{-\gamma}} - {1/2}.
\end{equation}
For satellites,
\begin{equation}
 \langle N_{\mathrm s}(M_\star|M)\rangle \equiv \int_{M_\star}^\infty \phi_{\mathrm s}(M_\star^\prime) \d \ln M_\star^\prime =  \langle N_{\mathrm c}(M_\star|M)\rangle \left({f^{-1}_{\mathrm SHMR}(M_\star) \over M_{\mathrm sat}}\right)^{\alpha_{\mathrm sat}} \exp\left(- {M_{\mathrm cut} \over f^{-1}_{\mathrm SHMR}(M_\star)} \right),
\end{equation}
where
\begin{equation}
 {M_{\mathrm sat} \over 10^{12} M_\odot} = B_{\mathrm sat} \left({f^{-1}_{\mathrm SHMR}(M_\star) \over 10^{12} M_\odot}\right)^{\beta_{\mathrm sat}},
\end{equation}
and
\begin{equation}
 {M_{\mathrm cut} \over 10^{12} M_\odot} = B_{\mathrm cut} \left({f^{-1}_{\mathrm SHMR}(M_\star) \over 10^{12} M_\odot}\right)^{\beta_{\mathrm cut}}.
\end{equation}
By default, parameter values are taken from the fit of \cite{leauthaud_new_2011}, specifically their {\normalfont \ttfamily SIG\_MOD1} method for their $z_1$ sample. These default values, and the \glc\ input parameters which can be used to adjust them are shown in Table~\ref{table:Behroozi2010FitParameters}. This method assumes that $P_{\mathrm s}(N|M_\star,M;\delta \ln M_\star)$ is a Poisson distribution while $P_{\mathrm c}(N|M_\star,M;\delta \ln M_\star)$ has a Bernoulli distribution, with each distribution's free parameter fixed by requiring
\begin{equation}
 \phi(M_\star;M) \delta \ln M_\star = \sum_{N=0}^\infty N P(N|M_\star,M;\delta \ln M_\star)
\end{equation}

\begin{table}
\caption{Parameters of the \cite{behroozi_comprehensive_2010} conditional stellar mass function model, along with their default values and the corresponding \glc\ input parameters.}
\label{table:Behroozi2010FitParameters}
\begin{center}
\begin{tabular}{lr@{.}ll}
\hline
{\normalfont \bfseries Parameter} & \multicolumn{2}{c}{{\normalfont \bfseries Default}} & {\normalfont \bfseries \glc\ name} \\
\hline
$\alpha_{\mathrm sat}$& 1&0& {\normalfont \ttfamily [conditionalStellarMassFunctionBehrooziAlphaSatellite]} \\
$\log_{10} M_1$& 12&520& {\normalfont \ttfamily [conditionalStellarMassFunctionBehrooziLog10M1]} \\
$\log_{10} M_{\star,0}$& 10&916& {\normalfont \ttfamily [conditionalStellarMassFunctionBehrooziLog10Mstar0]} \\
$\beta$& 0&457& {\normalfont \ttfamily [conditionalStellarMassFunctionBehrooziBeta]} \\
$\delta$& 0&5666& {\normalfont \ttfamily [conditionalStellarMassFunctionBehrooziDelta]} \\
$\gamma$& 1&53& {\normalfont \ttfamily [conditionalStellarMassFunctionBehrooziGamma]} \\
$\sigma_{\log M_\star}$& 0&206& {\normalfont \ttfamily [conditionalStellarMassFunctionBehrooziSigmaLogMstar]} \\
$B_{\mathrm cut}$& 1&47& {\normalfont \ttfamily [conditionalStellarMassFunctionBehrooziBCut]} \\
$B_{\mathrm sat}$& 10&62& {\normalfont \ttfamily [conditionalStellarMassFunctionBehrooziBSatellite]} \\
$\beta_{\mathrm cut}$& $-$0&13& {\normalfont \ttfamily [conditionalStellarMassFunctionBehrooziBetaCut]} \\
$\beta_{\mathrm sat}$& 0&859& {\normalfont \ttfamily [conditionalStellarMassFunctionBehrooziBetaCut]} \\
\hline
\end{tabular}
\end{center}
\end{table}

\section{Tree Timing}\label{sec:TreeTimingFile}

Estimates of the time taken to process a merger tree are used in some halo sampling rate functions and may in future be used in load balancing algorithms. \glc\ implements the following calculations of tree processing times, which can be selected via the {\normalfont \ttfamily [timePerTreeMethod]} input parameter.

\subsection{File Method}

Currently the only option, and selected using {\normalfont \ttfamily [timePerTreeMethod]}$=${\normalfont \ttfamily file}, this method assumes that the time taken to process a tree is given by
\begin{equation}
 \log_{10} [ \tau_{\mathrm tree}(M)] = \sum_{i=0}^2 C_i (\log_{10} M)^i,
\end{equation}
where $M$ is the root mass of the tree and the coefficients $C_i$ are read from a file, the name of which is specified via the {\normalfont \ttfamily [timePerTreeFitFileName]} parameter. This file should be an XML document with the structure:
\begin{verbatim}
<timing>
 <fit>
   <coefficient>-0.73</coefficient>
   <coefficient>-0.20</coefficient>
   <coefficient>0.03</coefficient>
 </fit>
</timing>
\end{verbatim}
where the array of coefficients give the values $C_0$, $C_1$ and $C_2$.

Note that, if \glc\ is run with {\normalfont \ttfamily [metaCollectTimingData]}$=${\normalfont \ttfamily true}, then it will output measures of tree processing time to the output file (see \S\ref{sec:MetaTreeTimingProfiler}). The analysis script {\normalfont \ttfamily scripts/analysis/treeTiming.pl} can be used to extract tree timing data from such an output file and output fitting coefficients in the above format. It is used as follows:
\begin{verbatim}
 treeTiming.pl <modelFile> [options.....]
\end{verbatim}
where {\normalfont \ttfamily <modelFile>} is the name of the \glc\ output file to analyze. The following options can be specified:
\begin{description}
 \item [{\normalfont \ttfamily accumulate}] If present, this argument will cause new timing data from the {\normalfont \ttfamily <modelFile>} is be accumulated with any timing data already present in the output file (which must be specified in this case). The fit is recomputed from the totallity of the accumulated data;
 \item [{\normalfont \ttfamily outputFile}] If present, the timing data for individual trees together with the fitting coefficiencts will be output to the specified file;
 \item [{\normalfont \ttfamily maxPoints}] When accumulating trees to the output file, this paramter, if present, will limit the number of trees stored in the file to the given value. The oldest trees added to the file will be dropped first;
 \item [{\normalfont \ttfamily plotFile}] If present, a plot of the tree timing as a function of halo mass, together with the fitting function, will be output to the specified file.
\end{description}
Note that the output file will contain both the fitting coefficients in the format described above and, additionally, a list of tree root masses and processing times (necessary if you later want to append trees from another run to this file).
