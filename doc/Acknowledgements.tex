\chapter{Acknowledgements}

In addition to the tools and libraries required to compile and run \glc, development of \glc\ has benefitted from extensive use of the following: \href{http://www.gnu.org/software/octave/}{{\normalfont \scshape GNU Octave}}, \href{http://maxima.sourceforge.net/}{{\normalfont \scshape Maxima}}, \href{http://edu.kde.org/cantor/}{{\normalfont \scshape Cantor}}, \href{http://kile.sourceforge.net/}{{\normalfont \scshape Kile}}, \href{http://www.gnu.org/software/emacs/}{{\normalfont \scshape Emacs}} and \href{http://valgrind.org/}{{\normalfont \scshape Valgrind}}. We are grateful to the members of the {\normalfont \scshape GNU Fortran} mailing list for invaluable discussions and fixes for compiler problems. We thank John Burkardt for making available the {\normalfont \scshape Bivar} algorithm for performing interpolation on data irregularly spaced on a 2D plane and Dima Verner for making available codes to compute various atomic data for astrophysics. Chris Power provided instructions for installing \glc\ under Mac OS X. The community of \glc\ users\footnote{In particular, Christoph Behrens, Stephanie D\"orschner, Jianling Gan, Markus Haider, Harald H\"oller, Eve Kovacs, Ting-Wen Lan, Adrian Pope, Luiz Felippe Rodrigues, Sergio Sanes, Martin White and Liyan Xu.} have provided invaluable feedback and bug reports. Gian Luigi Granato and Laura Silva kindly provided modifications to their \href{http://adlibitum.oat.ts.astro.it/silva/grasil/grasil.html}{\normalfont \scshape Grasil} code to allow it to read \glc\ outputs.
