\chapter{Running Galacticus}

\section{Configuration File}\label{sec:ConfigFile}\index{galacticusConfig.xml@{\tt galacticusConfig.xml}}\index{configuration}

The file {\tt galacticusConfig.xml}, is present, is used to configure \glc\ and provide useful information. It should have the following structure:
\begin{verbatim}
<config>
  <contact>
    <name>My Name</name>
    <email>me@ivory.towers.edu</email>
  </contact>
  <email>
    <host>
      <name>myComputerHostName</name>
      <method>smtp</method>
      <host>smtp-server.ivory.towers.edu</host>
      <user>myUserName</user>
      <passwordFrom>kdewallet</passwordFrom>
    </host>
    <host>
      <name>default</name>
      <method>sendmail</method>
    </host>
  </email>
</config>
\end{verbatim}
The name and e-mail address in the {\tt contact} section will be stored in any \glc\ models run---this helps track the provenance of the model. The {\tt email} section determines how e-mail will be sent. Within this section, you can place one or more {\tt host} elements, the {\tt name} element of which specifies the host name of the computer to which these rules apply (the {\tt default} host is used if no other match is found). For each host, the {\tt method} element specifies how e-mail should be sent, either by {\tt sendmail} or via {\tt smtp}. For SMTP transport (which currently supports SSL connections only), you must specify the {\tt host} SMTP server, {\tt user} name. The {\tt passwordFrom} element specifies how the password for the SMTP log in should be obtained. If set to {\tt input} then the user will be prompted for the password as needed. Alternatively, if you use the \href{http://www.kde.org/}{KDE} desktop and the \href{http://utils.kde.org/projects/kwalletmanager/}{KDEWallet} password manager, setting {\tt passwordFrom} to {\tt kdewallet} will cause the password to be stored in the KDE wallet and retrieved from there subsequently.

\section{Parameter Files}

\glc\ requires a file of parameters to be given as a command line argument. (In fact, it technically doesn't as all of the parameters will be assigned default values if no file is given, but in most cases you want to change some parameters, in which case you must specify them in a file.) The parameter file is an XML file (which makes it easy to manipulate and construct these files from within many languages, e.g. Perl) with the following structure:
\begin{verbatim}
 <parameters>
   <parameter>
     <name>parameter1name</name>
     <value>parameter1value</value>
   </parameter>
   .
   .
   .
 </parameters>
\end{verbatim}
Each {\tt parameter} element contains {\tt name} and {\tt value} elements which contain the parameter name and desired value respectively. The value can be a number, word(s) or an array of space-separated numbers. Parameters are used to control the values of numerical parameters and also to select methods and other options. If a parameter is not specified in the file a default value (hard coded into \glc) will be used instead. The default values have been chosen to produce a realistic model of galaxy formation, but may change as \glc\ evolves.

All parameter values (both those specified in this file and those set to default) used during a \glc\ run are output to the {\tt Parameters} group within the \glc\ output file. The script {\tt scripts/aux/Extract\_Parameter\_File.pl} will, if given a \glc\ output file, extract the parameters from it and output them into an XML file suitable for re-input into \glc.

\subsection{Generating Parameter Files}\index{parameters!generating}

Some scripts are provided which assist in the generation of parameter files. These are located in the {\tt scripts/parameters/} folder and are detailed below:
\begin{description}
\item [{\tt cosmologicalParametersMonteCarlo.pl}] This script will generate a set of cosmological parameters drawn at random from the WMAP-7 constraints \cite{komatsu_seven-year_2010}. It uses the covariance matrix (currently defined in {\tt data/Cosmological\_Parameters\_WMAP-7.xml}) to produce correlated random variables\footnote{Note that this does not capture the full details of the correlations between parameters, since it uses just the covariance matrix. For a more accurate calculation the full Monte Carlo Markov Chains used in the WMAP-7 parameter fitting should be used instead.}. The generated parameters are printed to standard output as \glc-compatible XML.
\end{description}

\section{Running Galacticus}

\glc\ is running using
\begin{verbatim}
 Galacticus.exe [<parameterFile>]
\end{verbatim}
where {\tt parameterFile} is the name of the file containing a list of parameter values for \glc. \glc\ will display messages indicating its progress as it runs (the verbosity can be controlled with the {\tt verbosityLevel} parameter).

\subsection{Restarting A Crashed Run}\label{sec:Restarting}

If \glc\ crashes, it can be useful to restart the calculation from just prior to the crash to speed the debugging process. \glc\ has functionality to store and retrieve the internal state of any modules and to recover this to permit such restarting. Currently, this is implemented with the {\tt build} method of merger tree construction, such that the internal state is stored prior to commencing the building of each tree, thereby allowing a calculation to be restarted with the tree that crashed. More general store/retrieve behavior is planned for future releases.

To cause \glc\ to periodically store its internal state include the following input parameter:
\begin{verbatim}
  <parameter>
    <name>stateFileRoot</name>
    <value>galacticusState</value>
  </parameter>
\end{verbatim}
This will cause the internal state to be stored to files {\tt galacticusState.state} and {\tt galacticusState.fgsl.state} prior to commencing building each merger tree. Should a tree crash then replace this input parameter with:
\begin{verbatim}
  <parameter>
    <name>stateRetrieveFileRoot</name>
    <value>galacticusState</value>
  </parameter>
  <parameter>
    <name>mergerTreeBuildTreesBeginAtTree</name>
    <value>N</value>
  </parameter>
\end{verbatim}
where {\tt N} is the number of the tree that crashed. This will cause calculations to begin with tree {\tt N} and for the internal state to be recovered from the above mentioned files. The resulting tree and all galaxy formation calculations should therefore proceed just as in the original run (and so create the same crash condition).

\subsection{Running Grids of Models}\label{sec:RunningGrids}

You can easily write your own scripts to generate parameter files and run \glc\ on these files. An example of such a script is {\tt scripts/aux/Run\_Galacticus.pl}. This script will loop over a sequence of parameter values, generate appropriate parameter files, run \glc\ using those parameters and analyze the results. To run the script simply enter:
\begin{verbatim}
 ./scripts/aux/Run_Galacticus.pl <runFile>
\end{verbatim}
This will launch a single instance of the script. Multiple instances can be launched and will share the work load (i.e. they will not attempt to run a model which another instance is already running or has finished). The {\tt runFile} is an XML file with the following structure:
\begin{verbatim}
<parameterGrid>
 <modelRootDirectory>models.new</modelRootDirectory>
 <baseParameters>newBestParametersQuick.xml</baseParameters>
 <parameters>
  <label>modelLabel</label>
  <parameter>
   <name>stabilityThresholdStellar</name>
   <value>1.1</value>
   <value>0.9</value>
  </parameter>
 </parameters>
 <parameters>
  <parameter>
   <name>stabilityThresholdGaseous</name>
   <value>1.1</value>
   <value>0.9</value>
  </parameter>
  <parameter>
   <name>imfSalpeterYieldInstantaneous</name>
   <value>0.02</value>
   <value>0.04</value>
  </parameter>
  <parameter>
   <name>starFormationTimescaleDisksMethod</name>
   <value>Kennicutt-Schmidt
    <parameter>
     <name>starFormationKennicuttSchmidtTruncate</name>
     <value>true</value>
     <value>false</value>
    </parameter>
   </value>
   <value>dynamical time</value>
  </parameter>
 </parameters>
</parameterGrid>
\end{verbatim}
Each {\tt parameters} block contains a list of parameters following the format used in standard \glc\ parameter files, with the difference that each parameter can have multiple {\tt values}. A model will be run for all possible combinations of these values. Additionally, any {\tt value} element may contain further {\tt parameter} elements. All possible values of these parameters will be looped over when, and only when, the appropriate value of the containing parameter is being used. For example, in the above example, models will be run with {\tt [starFormationKennicuttSchmidtTruncate]}$=${\tt true} and {\tt false} only when {\tt [starFormationTimescaleDisksMethod]}$=${\tt Kennicutt-Schmidt} and not when {\tt [starFormationTimescaleDisksMethod]}$=${\tt dynamical time}.

By default, each model is output into a sequentially numbered directory within the {\tt ./models} directory. By default, these directories have the prefix {\tt galacticus}. This can be changed by including a {\tt label} element inside a {\tt parameters} block, in which case the content of the {\tt label} element will be used as the prefix. This root directory can be modfified by the optional {\tt modelRootDirectory} element. Additionally, a set of base parameters can be read from a file specified by the {\tt baseParameters} file---these will be read before each model is run and before any variations in parameters for the specific model are applied. As such, it defines the default model around which parameter variations occur. Additional options that may be present in the file (as elements within the {\tt parameterGrid} element) are:
\begin{description}
\item[{\tt doAnalysis}]If set to ``no'' then no analysis scripts will be run on completed models, otherwise, they will be;
\item[{\tt emailReport}] If set to ``yes'' a report will be e-mailed to the address specified in {\tt galacticusOptions.xml} when a model fails. Otherwise, the report will be written to standard output instead.
\end{description}

In addition to the {\tt galacticus.hdf5} output file, each model directory will contain a file {\tt newParameters.xml} which contains the parameters used to run the model and {\tt galacticus.log} which contains any output from \glc\ during the run.

If present, the file {\tt galacticusConfig.xml}, described in \S\ref{sec:ConfigFile}, is parsed for configuration options. If the {\tt contact} element is present, the listed name and e-mail address will be used to determine who should receive error reports should a model crash. The error report will contain the host name of the computer running the model, the location of the model output and the log file (which may be incomplete if output is being buffered). Additionally, any core file produced will be stored in the model directory for later perusal, and the state files (see \S\ref{sec:Restarting}) for the run can also be found in the model directory.

\subsection{Analysis of Models}

The {\tt Run\_Galacticus.pl} script will automatically run {\tt scripts/analysis/Galacticus\_Compute\_Fit.pl} on each model to generate plots and fitting data. This script, which can also be running manually using
\begin{verbatim}
 ./scripts/analysis/Galacticus_Compute_Fit.pl <galacticusFile> <outputDirectory>
\end{verbatim}
where {\tt galacticusFile} is the name of the \glc\ output file to analyze and {\tt outputDirectory} is the directory into which plots and fitting data should be placed, reads the file {\tt data/Galacticus\_Compute\_Fit\_Analyses.xml} which has the following structure:
\begin{verbatim}
 <analyses>
  <analysis>
    <script>scripts/plotting/Plot_HI_Mass_Function.pl</script>
    <weight>1.0</weight>
  </analysis>
  <analysis>
    <script>scripts/plotting/Plot_K_Luminosity_Function.pl</script>
    <weight>1.0</weight>
  </analysis>
  .
  .
  .
 </analyses>
\end{verbatim}
Each {\tt analysis} element contains the name of a script to run to perform some analysis and a weight to be given to the results of this analysis when combining results to get a net goodness of fit. Each script listed will be run and is expected to have accept arguments of the form:
\begin{verbatim}
 My_Analysis_Script.pl <galacticusFile> <outputDirectory> <showFit>
\end{verbatim}
where the {\tt showFit} argument can be 0 or 1 and, if set to 1, the script should output an XML chunk to standard output giving details of its fitting analysis. This chunk should have the form:
\begin{verbatim}
 <galacticusFit>
   <name>Description of this analysis</name>
   <chiSquared>24.5</chiSquared>
   <degreesOfFreedom>19</degreesOfFreedom>
   <fileName>Output_File_Name.pdf</fileName>
 </galacticusFit>
\end{verbatim}
where {\tt chiSquared} and {\tt degreesOfFreedom} are the fitting results. All such data returned from fitting scripts will be collated by {\tt Galacticus\_Compute\_Fit.pl}, augmented with the weight value and the net goodness of fit determined. All of this information is then output to {\tt galacticusFits.xml} in the selected output directory.

\subsection{Running Models in ``Embarrassingly Parallel'' Mode}

While \glc\ is parallelized via OpenMP it is also possible to split a given model across several ``worker'' CPUs on one or more computers. The trees to be processed will be shared between these workers and the results can be later recombined. To use this ``poor man's'' parallelization, add the following to a model parameter file:
\begin{verbatim}
  <parameter>
    <name>treeEvolveWorkerCount</name>
    <value>N</value>
  </parameter>
  <parameter>
    <name>treeEvolveWorkerNumber</name>
    <value>i</value>
  </parameter>
\end{verbatim}
where {\tt N} is the total number of workers to be used and {\tt i} is the number of this worker (ranging from 1 to {\tt N}). Once all workers have finished, their outputs can (if required) be combined into a single output file using the {\tt Merge\_Models.pl} script as follows:
\begin{verbatim}
./scripts/aux/Merge_Models.pl <model1> <model2> .... <modelOutput>
\end{verbatim}
where {\tt model1} etc. are the names of the various output files and {\tt modelOutput} is the file into which the combined results should be placed. The {\tt Merge\_Models.pl} script will combine all merger trees into the output file and will additionally cumulate any data in the {\tt globalHistory} groups in these files.

\section{Additional Codes}

The \glc\ code base can be used for other calculations. Some examples of such usage (and which are sufficiently useful in their own right) are included and are detailed in this section.

\subsection{{\tt Halo\_Mass\_Functions}}

The {\tt Halo\_Mass\_Functions} code will generate an HDF5 output file which contains a variety of measures of the dark matter halo mass function tabulated as a function of mass and at a variety of redshifts. The code is built and run as follows:
\begin{verbatim}
make Halo_Mass_Function.exe
Halo_Mass_Function.exe <parameterFile> <outputFile>
\end{verbatim}
where {\tt parameterFile} is a file of parameters in \glc's usual XML format and {\tt outputFile} is the name of the file to which the halo mass function data will be written. The parameter file can specify any parameters needed for computing the mass function (they will be set to default values in cases where a paramter is not included). The redshifts at which to output halo mass functions are given by the {\tt [outputRedshifts]} parameter. In addition to the usual \glc\ parameters three additional parameters control behavior:
\begin{description}
\item [{\tt [haloMassFunctionsMassMinimum]}] The lowest mass halo (in units of $M_\odot$) at which to tabulate;
\item [{\tt [haloMassFunctionsMassMaximum]}] The highest mass halo (in units of $M_\odot$) at which to tabulate;
\item [{\tt [haloMassFunctionsPointsPerDecade]}] The number of points per decade of halo mass at which to tabulate.
\end{description}
The output file has the following structure:
\begin{verbatim}
+-> Outputs
|   |
|   +-> outputCharacteristicMass      [dataset]
|   |
|   +-> outputCriticalOverdensities   [dataset]
|   |
|   +-> outputExpansionFactor         [dataset]
|   |
|   +-> outputGrowthFactors           [dataset]
|   |
|   +-> outputRedshift                [dataset]
|   |
|   +-> outputTime                    [dataset]
|   |
|   +-> outputVirialDensityContrast   [dataset]
|    
+-> Parameters
|   |
|   +-> parameter1                    [attribute]
|   |
|   +-> parameterN                    [attribute]
|    
+-> haloMassFunctions
    |
    +-> haloBias                      [dataset]
    |
    +-> haloMass                      [dataset]
    |
    +-> haloMassFractionCumulative    [dataset]
    |
    +-> haloMassFunctionCumulative    [dataset]
    |
    +-> haloMassFunctionLnM           [dataset]
    |
    +-> haloMassFunctionM             [dataset]
    |
    +-> haloNu                        [dataset]
    |
    +-> haloSigma                     [dataset]
    |
    +-> haloVirialRadius              [dataset]
    |
    +-> haloVirialTemperature         [dataset]
    |
    +-> haloVirialVelocity            [dataset]
\end{verbatim}
The {\tt Parameters} group contains attributes giving the values of all used parameters (just as in a \glc\ output file). The {\tt Outputs} group contains datasets which give global properties at each requested output time as follows:
\begin{description}
\item [{\tt outputCharacteristicMass}] The characteristic mass scale (in units of $M_\odot$), $M_*$, at which $\sigma(M)=\delta_{\rm c}(z)$;
\item [{\tt outputCriticalOverdensities}] The critical overdensity for collapse of halos, $\delta_{\rm c}$;
\item [{\tt outputExpansionFactor}] The expansion factor;
\item [{\tt outputGrowthFactors}] The linear growth factor;
\item [{\tt outputRedshift}] The redshift;
\item [{\tt outputTime}] The cosmic time (in units of Gyr);
\item [{\tt outputVirialDensityContrast}] The virial density contrast of halos.
\end{description}
The {\tt haloMassFunctions} group contains datasets which list the properties of halos as a function of mass at each requested output time as follows:
\begin{description}
\item [{\tt haloBias}] The large scale linear theory bias of the halo;
\item [{\tt haloMass}] The mass of the halo (in $M_\odot$);
\item [{\tt haloMassFractionCumulative}] The mass fraction in halos above the current halo mass;
\item [{\tt haloMassFunctionCumulative}] The cumulative number of halos per unit volume above the current halo mass (in units of Mpc$^{-3}$);
\item [{\tt haloMassFunctionLnM}] The halo mass function per logarithmic halo mass (in units of Mpc$^{-3}$);
\item [{\tt haloMassFunctionM}] The halo mass function per logarithmic halo mass (in units of Mpc$^{-3} M_\odot^{-1}$);
\item [{\tt haloNu}] The peak height of the halo, $\nu = \delta_{\rm c}/\sigma(M)$;
\item [{\tt haloSigma}] The root-variance of the mass field smoothed in top-hat spheres;
\item [{\tt haloVirialRadius}] The virial radius (in units of Mpc) of the current halo mass;
\item [{\tt haloVirialTemperature}] The virial temperature (in units of Kelvin) of the current halo mass;
\item [{\tt haloVirialVelocity}] The virial velocity (in units of km/s) of the current halo mass;
\end{description}
Dimensionful datasets have an {\tt unitsInSI} attribute that gives their units in the SI system.
