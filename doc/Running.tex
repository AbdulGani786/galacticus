\chapter{Running Galacticus}

If you have not yet installed \glc\ you should follow the instructions \href{https://github.com/galacticusorg/galacticus/wiki#how-do-i-install-and-use-galacticus}{here} to do so.

\section{Setting the Environment}

Before running \glc\ you will need to set two environment variables which specify where the \glc\ source code and datasets can be found. First, the environment variable {\normalfont \ttfamily GALACTICUS\_EXEC\_PATH} should be set to the full path to the build directory\index{path!galacticus root@{\glc\ root}}. Second, the environment variable {\normalfont \ttfamily GALACTICUS\_DATA\_PATH} should be set to the full path to the {\normalfont \ttfamily datasets} directory which was created when you installed \glc.

\section{Running Galacticus}

\glc\ is run using
\begin{verbatim}
 Galacticus.exe <parameterFile>
\end{verbatim}
where {\normalfont \ttfamily parameterFile} is the name of the file containing parameter settings for \glc. \glc\ will display messages indicating its progress as it runs.

A simple example, useful to check that everything is working as expected for you, is to do:
\begin{verbatim}
 Galacticus.exe $GALACTICUS_EXEC_PATH/parameters/quickTest.xml
\end{verbatim}
This will run a small model. You should expect to see output which looks something like this:
\begin{verbatim}
              ##                                     
   ####        #                  #                  
  #   #        #             #                       
 #       ###   #  ###   ### ###  ##   ### ## ##   ## 
 #       #  #  #  #  # #  #  #    #  #  #  #  #  #   
 #   ###  ###  #   ### #     #    #  #     #  #   #  
  #   #  #  #  #  #  # #     #    #  #     #  #    # 
   ####  #### ### ####  ###   ## ###  ###   #### ##  

  2009, 2010, 2011, 2012, 2013, 2014, 2015, 2016,
  2017, 2018, 2019, 2020, 2021, 2022
   - Andrew Benson

MM: -> Begin task: merger tree evolution
 0:     -> Evolving tree number 1 {1}
 9:     -> Evolving tree number 3 {1}
 7:     -> Evolving tree number 4 {1}
 3:     -> Evolving tree number 2 {1}
 0:         Output tree data at t=  13.79 Gyr
 1:     -> Evolving tree number 5 {1}
11:     -> Evolving tree number 6 {1}
 0:     <- Finished tree
 9:         Output tree data at t=  13.79 Gyr
 9:     <- Finished tree
 3:         Output tree data at t=  13.79 Gyr
 3:     <- Finished tree
 7:         Output tree data at t=  13.79 Gyr
 7:     <- Finished tree
 1:         Output tree data at t=  13.79 Gyr
 1:     <- Finished tree
11:         Output tree data at t=  13.79 Gyr
11:     <- Finished tree
MM: <- Done task: merger tree evolution
\end{verbatim}
In this simple model, \glc\ is told to build six merger trees, and form galaxies within them, outputting the results at $z=0$. After the initial ``\glc'' banner is displayed there are several messages shown. Each line begins with either ``{\normalfont \ttfamily MM:}'', or a number. \glc\ runs in parallel using the available cores on your computer---these prefixes on each line tell you which parallel thread is sending the message. ``{\normalfont \ttfamily MM:}'' means that the message is from the main thread (and that \glc\ is currently running in a serial part of the code), while a numerical prefixes gives the number of that parallel thread reporting the message (starting from 0).

The first message, from the main thread, tells you that the ``merger tree evolution'' task has begun---in this task, \glc\ forms galaxies within a set of merger trees. After that initial message you'll see that \glc\ enters parallel calculation and messages start being reported from each parallel thread. The threads report when they begin evolving a merger tree (in this case merger trees are numbered consecutively, starting from 1), when they are outputting the data from that tree (at a time of $13.79$~Gyr, which corresponds to $z=0$ in this model), and when they have finished the tree. Once all trees are finished there is a final message from the main thread (parallel calculation is over at this point) indicating that the merger tree evolution task is finished. \glc\ then exits, since it has finished all of the work we asked it to do.

The results will have been written to an output file named {\normalfont \ttfamily galacticus.hdf5}, which you can see by doing:
\begin{verbatim}
ls -l galacticus.hdf5
\end{verbatim}
which will show something similar to:
\begin{verbatim}
-rw-r--r-- 1 abenson users 568601 Oct 15 20:58 galacticus.hdf5
\end{verbatim}
\glc\ outputs use the \href{http://www.hdfgroup.org/HDF5/}{HDF5} format---libraries for accessing HDF5 files exist in all major languages, including Python\footnote{\href{https://www.h5py.org/}{{\normalfont \scshape h5py}}}. We will explore the structure of the output file in more detail in \S\ref{sec:outputFile}, but for now we'll simply explore some of the key features using the command line.

The content of the output file can be explored using the ``{\normalfont \ttfamily h5ls}'' tool, for example:
\begin{verbatim}
$ h5ls galacticus.hdf5
Build                    Group
Outputs                  Group
Parameters               Group
Version                  Group
\end{verbatim}
The file contains several groups---the data corresponding to the galaxies that were formed can be found in the {\normalfont \ttfamily Outputs} group:
\begin{verbatim}
$ h5ls galacticus.hdf5/Outputs
Output1                  Group
\end{verbatim}
In this case there is only a single output, corresponding to $z=0$, we can explore that group using:
\begin{verbatim}
$ h5ls galacticus.hdf5/Outputs/Output1
mergerTreeCount          Dataset {6/Inf}
mergerTreeIndex          Dataset {6/Inf}
mergerTreeStartIndex     Dataset {6/Inf}
mergerTreeWeight         Dataset {6/Inf}
nodeData                 Group
\end{verbatim}
In this output group we find some datasets which give information on the merger trees that were used, plus another group, ``{\normalfont \ttfamily nodeData}'', inside of which the galaxy data is stored:
\begin{verbatim}
$ h5ls galacticus.hdf5/Outputs/Output1/nodeData
basicMass                Dataset {7/Inf}
basicTimeLastIsolated    Dataset {7/Inf}
blackHoleCount           Dataset {7/Inf}
blackHoleMass            Dataset {7/Inf}
blackHoleSpin            Dataset {7/Inf}
darkMatterProfileScale   Dataset {7/Inf}
diskAbundancesGasMetals  Dataset {7/Inf}
diskAbundancesStellarMetals Dataset {7/Inf}
diskAngularMomentum      Dataset {7/Inf}
diskMassGas              Dataset {7/Inf}
diskMassStellar          Dataset {7/Inf}
diskRadius               Dataset {7/Inf}
diskVelocity             Dataset {7/Inf}
hotHaloAbundancesMetals  Dataset {7/Inf}
hotHaloAngularMomentum   Dataset {7/Inf}
hotHaloMass              Dataset {7/Inf}
hotHaloOuterRadius       Dataset {7/Inf}
hotHaloOutflowedAbundancesMetals Dataset {7/Inf}
hotHaloOutflowedAngularMomentum Dataset {7/Inf}
hotHaloOutflowedMass     Dataset {7/Inf}
hotHaloUnaccretedMass    Dataset {7/Inf}
nodeIndex                Dataset {7/Inf}
nodeIsIsolated           Dataset {7/Inf}
parentIndex              Dataset {7/Inf}
satelliteBoundMass       Dataset {7/Inf}
satelliteIndex           Dataset {7/Inf}
satelliteMergeTime       Dataset {7/Inf}
siblingIndex             Dataset {7/Inf}
spheroidAbundancesGasMetals Dataset {7/Inf}
spheroidAbundancesStellarMetals Dataset {7/Inf}
spheroidAngularMomentum  Dataset {7/Inf}
spheroidMassGas          Dataset {7/Inf}
spheroidMassStellar      Dataset {7/Inf}
spheroidRadius           Dataset {7/Inf}
spheroidVelocity         Dataset {7/Inf}
spinSpin                 Dataset {7/Inf}
\end{verbatim}
Each dataset here is an array containing the named property of each galaxy formed in the model. In this tiny example model only 7 galaxies were formed. To see the values of each property we can do:
\begin{verbatim}
$ h5ls -d galacticus.hdf5/Outputs/Output1/nodeData/diskMassStellar
diskMassStellar          Dataset {7/Inf}
    Data:
        (0) 0, 719592.675733525, 0, 36664762.1643418, 75533178.4182319, 0, 764729109.215361
\end{verbatim}
which lists the mass of stars in each galaxy disk (in units of $\mathrm{M}_\odot$) (note that some of them are zero---these halos in the merger tree either formed no galaxy, or formed a galaxy with no disk component).

These data can be extracted and analyzed using any software or language that supports reading HDF5 files.
