\chapter{Physical Implementations}

\section{Accretion of Gas into Halos}\label{sec:AccretionBaryonic}\index{accretion!baryonic}

The accretion rate of gas from the \IGM\ into a dark matter halo is expected to depend on (at least) the rate at which that halo mass is growing, the depth of its potential well and the thermodynamical properties of the accreting gas. \glc\ implements the following calculations of gas accretion from the \IGM, which can be selected via the {\tt accretionHalosMethod} input parameter.

\subsection{Simple Method}

Currently the only option, and selected using {\tt accretionHalosMethod}$=${\tt simple}, this method sets the accretion rate of baryons into a halo to be:
\begin{equation}
 \dot{M}_{\rm accretion} = \left\{ \begin{array}{ll} (\Omega_{\rm b}/\Omega_0) \dot{M}_{\rm halo} & \hbox{ if } V_{\rm virial} > V_{\rm reionization} \hbox{ or } z > z_{\rm reionization} \\ 0 & \hbox{ otherwise,}\end{array} \right.
\end{equation}
where $z_{\rm reionization}=${\tt [reionizationSuppressionRedshift]} is the redshift at which the Universe is reionized and $V_{\rm reionization}=${\tt [reionizationSuppressionVelocity]} is the virial velocity below which accretion is suppressed after reionization. Setting $V_{\rm reionization}$ to zero will effectively switch off the effects of reionization on the accretion of baryonics. This algorithm attempts to offer a simple prescription for the effects of reionization and has been explored by multiple authors (e.g. \citealt{benson_effects_2002}). In particular, \cite{font_modelingmilky_2010} show that it produces results in good agreement with more elaborate treatments of reionization. For halos below the accretion threshold, any accretion rate that would have otherwise occurred is instead placed into the ``failed'' accretion rate. For halos which can accrete, and which have some mass in their ``failed'' reservoir, that mass will be added to the regular accretion rate at a rate equal to the mass of the ``failed'' reservoir times the specific growth rate of the halo. The gas accreted is assumed to be from a pristine \IGM\ and so has zero abundances. Molecular abundances are computed from the atomic ionization state functions (see \S\ref{sec:IonizationStateMethod}).

\section{Background Cosmology}\index{cosmology}

The background cosmology describes the evolution of an isotropic, homoegeneous Universe within which our calculations are carried out. For the purposes of \glc, the background cosmology is used to relate expansion factor/redshift to cosmic time and to compute the density of various components (e.g. dark matter, dark energy, etc.) at different epochs. Background cosmological models are specified via the {\tt cosmologyMethod}, and the physics that must be implemented for each cosmological model is describe in more detail in \S\ref{sec:CosmologyMethods}. Currently implemented cosmological models are as follows.

\subsection{Matter + Lambda}

Selected with {\tt cosmologyMethod}$=${\tt matter + lambda}, in this implementation cosmological relations are computed assuming a universe that contains only collisionless matter and a cosmological constant.


\section{Circumnuclear Accretion Disks}\label{sec:CircumnuclearDisks}\index{accretion disks}\index{accretion!disk}

Circumnuclear accretion disks surrounding supermassive black holes at the centers of galaxies influence the evolution of both the black hole (via accretion rates of mass and angular momentum and possibly by extracting rotational energy from the black hole) and the surrounding galaxy if they lead to energetic outflows (e.g. jets) from the nuclear region. Accretion disk type is specified via the {\tt accretionDisksMethod}, and the physics that must be implemented for each accretion disk type is describe in more detail in \S\ref{sec:AccretionDisks}. Current implementations of accretion disks are as follows.

\subsection{Shakura-Sunyaev Geometrically Thin, Radiatively Efficient Disks}

Selected with {\tt accretionDisksMethod}$=${\tt Shakura-Sunyaev}, this implementation assumes that accretion disks are always described by a radiatively efficient, geometrically thin accretion disk as described by \cite{shakura_black_1973}. The radiative efficiency of the flow is computed assuming that material falls into the black hole without further energy loss from the \ISCO, while the spin-up rate of the black hole is computed assuming that the material enters the black hole with the specific angular momentum of the \ISCO\ (i.e. there are no torques on the material once it begins to fall in from the \ISCO; \citealt{bardeen_kerr_1970}). For these thin disks, jet power is computed, using the expressions from \citeauthor{meier_association_2001}~(\citeyear{meier_association_2001}; his equations 4 and 5).

\subsection{Advection Dominated, Geometrically Thick, Radiatively Inefficient Flows (ADAFs)}

Selected with {\tt accretionDisksMethod}$=${\tt ADAF}, this implementation assumes that accretion is via an advection dominated accretion flow \citep{narayan_advection-dominated_1994} which is radiatively inefficient and geometrically thick. The radiative efficiency of the flow, which will be zero for a pure ADAF, can be set via the input parameter {\tt [adafRadiativeEfficiency]}. The spin up rate of the black hole and the jet power produced as material accretes into the black hole are computed using the method of \cite{benson_maximum_2009}. The energy of the accreted material can be set equal to the energy at infinity (as expected for a pure ADAF) or the energy at the \ISCO\ by use of the {\tt [adafEnergyOption]} parameter (set to {\tt pure ADAF} or {\tt ISCO} respectively). The ADAF structure is controlled by the adiabatic index, $\gamma$, and viscosity parameter, $\alpha$, which are specified via the {\tt [adafAdiabaticIndex]} and {\tt [adafViscosityOption]} input parameters respectively. The field-enhancing shear, $g$, is computed using $g=\exp(\omega \tau)$ if {\tt [adafFieldEnhanceType]} is set to ``exponential'' where $\omega$ is the frame-dragging frequency and $\tau$ is the smaller of the radial inflow and azimuthal velocity timescales. If  {\tt [adafFieldEnhanceType]} is set to ``linear'' then the alternative version, $g=1+\omega \tau$ is used instead. {\tt [adafViscosityOption]} may be set to ``{\tt fit}'', in which case the fitting function for $\alpha$ as a function of black hole spin is used:
\begin{eqnarray}
\alpha(j)=0.015+0.02 j^4 & \hbox{ if  }& g=\exp(\omega\tau) \hbox{ and } E=E_{\rm ISCO}, \\
\alpha(j)=0.025+0.08 j^4 & \hbox{ if } & g=1+\omega\tau \hbox{ and } E=E_{\rm ISCO}, \\
\alpha(j)=0.010+0.00 j^4 & \hbox{ if } & g=\exp(\omega\tau) \hbox{ and } E=1, \\
\alpha(j)=0.025+0.02 j^4 & \hbox{ if } & g=1+\omega\tau \hbox{ and } E=1.  
\end{eqnarray}

\subsection{``Switched'' Disks}

Selected with {\tt accretionDisksMethod}$=${\tt switched}, this method allows for accretion disks to switched between radiatively efficient (Shakura-Sunyaev) and inefficient (ADAF) modes. TWhich mode is used is determine by the accretion rate onto the disk:
\begin{itemize}
 \item Radiatively efficient accretion if $\dot{M}/\dot{M}_{\rm Eddington}>${\tt accretionRateThinDiskMinimum} and $\dot{M}/\dot{M}_{\rm Eddington}<${\tt accretionRateThinDiskMaximum};
 \item Radiatively inefficient accretion otherwise.
\end{itemize}
Both {\tt accretionRateThinDiskMinimum} and {\tt accretionRateThinDiskMaximum} are adjustable input parameters.

\section{Cold Dark Matter Structure Formation}\index{structure formation}\index{cold dark matter}

A variety of functions are used to describe structure formation in cold dark matter dominated universes. These are described below.

\subsection{Primordial Power Spectrum}\label{sec:PrimordialPowerSpectrum}\index{power spectrum!primordial}

The functional form of the primordial dark matter power spectrum is selected via the {\tt powerSpectrumMethod} parameter. The power spectrum is computed from the specified primordial power spectrum and the transfer function (see \S\ref{sec:TransferFunction}) and normalized to a value of $\sigma_8$ specified by {\tt [sigma\_8]}.

\subsubsection{(Running) Power Law Spectrum}

Selected via {\tt powerSpectrumMethod}$=${\tt power law}, this method implements a primordial power spectrum of the form:
\begin{equation}
 P(k) \propto k^{n_{\rm eff}(k)},
\end{equation}
where
\begin{equation}
 n_{\rm eff}(k) = n_{\rm s} + {1\over 2}{\d n \over \d \ln k} \ln \left( {k \over k_{\rm ref}} \right),
\end{equation}
where $n_{\rm s}=${\tt powerSpectrumIndex} is the power spectrum index at wavenumber $k_{\rm ref}=${\tt powerSpectrumReferenceWavenumber} and $\d n / \d \ln k=${\tt powerSpectrumRunning} describes the running of this index with wavenumber.

\subsection{Transfer Function}\label{sec:TransferFunction}\index{transfer function}

The functional form of the cold dark matter transfer function is selected via the {\tt transferFunctionMethod} parameter. The power spectrum is computed from the specified transfer function and the primordial power spectrum (see \S\ref{sec:PrimordialPowerSpectrum}) and normalized to a value of $\sigma_8$ specified by {\tt [sigma\_8]}.

\subsubsection{BBKS}

Selected with {\tt transferFunctionMethod}$=${\tt BBKS}, this method uses the fitting function of \cite{bardeen_statistics_1986} to compute the \CDM\ transfer function. The BBKS warm dark matter transfer function\index{transfer function!warm dark matter}\index{warm dark matter!transfer function} can be used by specifying the appropriate streaming length (in Mpc) via the {\tt [transferFunctionWDMFreeStreamingLength]} parameter.

\subsubsection{Eisenstein \& Hu}

Selected with {\tt transferFunctionMethod}$=${\tt Eisenstein + Hu}, this method uses the fitting function of \cite{eisenstein_power_1999} to compute the \CDM\ transfer function. It requires that the effective number of neutrino species be specified via the {\tt effectiveNumberNeutrinos} parameter and summed mass of all neutrino species (in eV) be specified via the {\tt summedNeutrinoMasses} parameter.

\subsubsection{{\sc CMBFast}}

Selected with {\tt transferFunctionMethod}$=${\tt CMBFast}, this method uses the {\sc CMBFast} code to compute the \CDM\ transfer function. It requires that the mass fraction of helium in the early Universe be specified via the {\tt Y\_He} parameter. {\sc CMBFast} will be downloaded and run if the transfer function needs to be computed. It will then be stored in a file for future reference.

\subsubsection{File}

Selected with {\tt transferFunctionMethod}$=${\tt file}, this method reads a tabulated transfer function from an XML file (specified via the {\tt transferFunctionFile} parameter), interpolating between tabulated points. The structure of the transfer function file is described in \S\ref{sec:TransferFunctionMethod}.

\subsection{Linear Growth Function}\index{linear growth}

The function describing the amplitude of linear perturbations is selected via the {\tt linearGrowthMethod} parameter.

\subsubsection{Simple}

Selected with {\tt linearGrowthMethod}$=${\tt simple}, this method calculates the growth of linear perturbations using standard perturbation theory in a Universe consisting of matter and a cosmological constant. Perturbations in the baryons are treated just as for dark matter (i.e. pressure forces are ignored), while perturbations in the radiation are assumed not to grow.

\subsection{Critical Overdensity}\index{density!critical}

The method used to compute the critical linear overdensity at which overdense regions virialize is selected via the {\tt criticalOverdensityMethod} parameter.

\subsubsection{Spherical Collapse (Matter + Cosmological Constant)}

Selected with {\tt criticalOverdensityMethod}$=${\tt spherical top hat} this method calculates critical overdensity using a spherical top-hat collapse model assuming a Universe which contains matter and a cosmological constant (see, for example, \citealt{percival_cosmological_2005}).

\subsection{Virial Density Contrast}\label{sec:VirialDensityConstrast}\index{density!virial}

The method used to compute the mean density contrast of virialized dark matter halos is selected via the {\tt virialDensityContrastMethod} parameter.

\subsubsection{Bryan \& Norman Fitting Function}

Selected with {\tt virialDensityContrastMethod}$=${\tt Bryan + Norman} this method calculates virial density contrast using the fitting functions given by \cite{bryan_statistical_1998}. As such, it is valid only for $\Omega_\Lambda=0$ or $\Omega_M+\Omega_\Lambda=1$ cosmologies and will abort on other cosmologies.

\subsubsection{Fixed}

Selected with {\tt virialDensityContrastMethod}$=${\tt fixed} this method uses a fixed virial density contrast of {\tt [virialDensityConstrastFixed]}, defined relative to {\tt critical density} and {\tt mean density} as specified by {\tt [virialDensityConstrastFixedType]}.

\subsubsection{Spherical Collapse (Matter + Cosmological Constant)}

Selected with {\tt virialDensityContrastMethod}$=${\tt spherical top hat} this method calculates virial density contrast using a spherical top-hat collapse model assuming a Universe which contains matter and a cosmological constant (see, for example, \citealt{percival_cosmological_2005}).

\subsection{Halo Bias}\index{halo bias}\index{dark matter halos!bias}\index{bias!halo}

The dark matter halo linear bias method is selected via the {\tt darkMatterHaloBiasMethod} parameter.

\subsubsection{Press-Schechter}

Selected with {\tt darkMatterHaloBiasMethod}$=${\tt Press-Schechter} this method uses a bias consistent with the halo mass function of \cite{press_formation_1974} (see \citep{mo_analytic_1996}).

\subsubsection{Sheth-Tormen}

Selected with {\tt darkMatterHaloBiasMethod}$=${\tt SMT} this method uses a bias consistent with the halo mass function of \cite{sheth_ellipsoidal_2001}.

\subsubsection{Tinker}

Selected with {\tt darkMatterHaloBiasMethod}$=${\tt Tinker2010} this method uses the functional form proposed by \cite{tinker_large_2010} to compute the halo bias. The bias is computed at the appropriate virial overdensity (see \S\ref{sec:VirialDensityConstrast}).

\subsection{Halo Mass Function}\index{halo mass function}\index{dark matter halos!mass function}

The dark matter halo mass function (i.e. the number of halos per unit volume per unit mass interval) is selected via the {\tt haloMassFunctionMethod} parameter.

\subsubsection{Press-Schechter}

Selected with {\tt haloMassFunctionMethod}$=${\tt Press-Schechter} this method uses the functional form proposed by \cite{press_formation_1974} to compute the halo mass function.

\subsubsection{Sheth-Tormen}

Selected with {\tt haloMassFunctionMethod}$=${\tt Sheth-Tormen} this method uses the functional form proposed by \cite{sheth_ellipsoidal_2001} to compute the halo mass function.

\subsubsection{Tinker}

Selected with {\tt haloMassFunctionMethod}$=${\tt Tinker2008} this method uses the functional form proposed by \cite{tinker_towardhalo_2008} to compute the halo mass function. The mass function is computed at the appropriate virial overdensity (see \S\ref{sec:VirialDensityConstrast}).

\section{Cooling of Gas Inside Halos}\index{cooling}

The cooling of gas within dark matter halos is controlled by a number of different algorithms which will be decribed below.

\subsection{Cooling Function}\index{cooling function}\index{cooling!cooling function}

The cooling function of gas, $\Lambda(\rho,T,{\bf Z})$, is implemented by the algorithm(s) selected using the {\tt coolingFunctionMethods} parameter. If more than one cooling function is specified, then the net cooling function is a sum over all of those selected.

\subsubsection{Atomic Collisional Ionization Equilibrium Using {\sc Cloudy}}

Selected using {\tt coolingFunctionMethods}$=${\tt atomic\_CIE\_Cloudy}, this method computes the cooling function using the {\sc Cloudy} code and under the assumption of collisional ionization equilibrium with no molecular contribution. Abundances are Solar, except for zero metallicity calculations which use {\sc Cloudy}'s ``primordial'' metallicity. The helium abundance for non-zero metallicity is scaled between primordial and Solar values linearly with metallicity. The {\sc Cloudy} code will be downloaded and run to compute the cooling function as needed, which will then be stored for future use. As this process is slow, a precomputed table is provided with \glc. If metallicities outside the range tabulated in this file are required it will be regenerated with an appropriate range.

\subsubsection{Collisional Ionization Eqiulibrium From File}

Selected using {\tt coolingFunctionMethods}$=${\tt CIE\_from\_file}, in this method the cooling function is read from a file specified by the {\tt coolingFunctionFile} parameter. The format of this file is specified in \S\ref{sec:CoolingFunctionMethods}. The cooling function is assumed to be computed under conditions of collisional ionization equilibrium and therefore to scale as $\rho^2$.

\subsubsection{CMB Compton Cooling}

Selected using {\tt coolingFunctionMethods}$=${\tt CMB\_Compton}, this method computes the cooling function due to Compton scattering off of \CMB\ photons:
\begin{equation}
\Lambda = {4 \sigma_{\rm T} {\rm a} {\rm k}_{\rm B} n_{\rm e } \over m_{\rm e} \clight} T_{\rm CMB}^4 \left( T - T_{\rm CMB} \right),
\end{equation}
where $\sigma_{\rm T}$ is the Thompson cross-section, $a$ is the radiation constant, ${\rm k}_{\rm B}$ is Boltzmann's constant, $n_{\rm e}$ is the number density of electrons, $m_{\rm e}$ is the electron mass, $\clight$ is the speed of light, $T_{\rm CMB}$ is the \CMB\ temperature at the current cosmic epoch and $T$ is the temperature of the gas. The electron density is computed from the selected ionization state method (see \S\ref{sec:IonizationStateMethod}).

\subsubsection{Molecular Hydrogen (Galli-Palla)}

Selected using {\tt coolingFunctionMethods}$=${\tt molecularHydrogenGalliPalla}, this method computes the cooling function due to molecular hydrogen using the results of \cite{galli_chemistry_1998}. For the H--H$_2$ cooling function, the fitting functions from \cite{galli_chemistry_1998} are used. For the H$_2^+$--e$^-$ and H--H$_2^+$ cooling functions fitting functions to the results plotted in  \cite{suchkov_cooling_1978} are used:
\begin{equation}
\log_{10}\left({\Lambda(T) \over \hbox{erg s}^{-1} \hbox{cm}^3}\right) = C_0 + C_1 \log_{10} \left({T\over\hbox{K}}\right) + C_2 \left[\log_{10} \left({T\over\hbox{K}}\right)\right]^2,
\label{eq:H2CoolingFunction}
\end{equation}
where the coefficients $C_{0-2}$ are given in Table~\ref{tb:H2CoolingFunctionCoefficients}.

\begin{table}
 \begin{center}
  \caption{Coefficients of H$_2^+$ cooling functions as appearing in the fitting function, eq.~\protect\ref{eq:H2CoolingFunction}.}
  \label{tb:H2CoolingFunctionCoefficients}
  \begin{tabular}{lrrr}
   \hline
   & \multicolumn{3}{c}{{\bf Coefficient}} \\
   {\bf Interaction} & \boldmath{$C_0$} & \boldmath{$C_1$} & \boldmath{$C_2$} \\
   \hline
   H$_2^+$--e$^-$ & -33.33 & 5.565 & -0.4675 \\
   H--H$_2^+$ & -35.28 & 5.862 & -0.5124 \\
   \hline
  \end{tabular}
 \end{center}
\end{table}


\subsection{Cooling Rate}\label{sec:CoolingRate}\index{cooling!rate}

The algorithm used to compute the rate at which gas drops out of the hot halo due to cooling is selected with the {\tt coolingRateMethod} parameter.

\subsubsection{White \& Frenk}

Selected with {\tt coolingRateMethod}$=${\tt White + Frenk}, this method computes the cooling rate using the expression given by \cite{white_galaxy_1991}, namely
\begin{equation}
\dot{M}_{\rm cool} = 4 \pi r_{\rm cool}^2 \rho(r_{\rm cool}) \dot{r}_{\rm cool},
\end{equation}
where $r_{\rm cool}$ is the cooling radius in the hot halo and $\rho(r)$ is the density profile of the hot halo.

\subsection{Cooling Radius}\index{cooling radius}\index{cooling!radius}

The algorithm used to compute the cooling radius is selected via the {\tt coolingRadiusMethod} parameter.

\subsubsection{Simple}

Selcted with {\tt coolingRadiusMethod}$=${\tt simple}, this method computes the cooling radius by seeking the radius at which the time available for cooling (see \S\ref{sec:TimeAvailableCooling}) equals the cooling time (see \S\ref{sec:CoolingTime}). The growth rate is determined consistently based on the slope of the density profile, the density dependence of the cooling function and the rate at which the time available for cooling is increasing. This method assumes that the cooling time is a monotonic function of radius.

\subsection{Cooling Time}\label{sec:CoolingTime}\index{cooling time}\index{cooling!time}

The algorithm used to compute the time taken for gas to cool (i.e. the cooling time) is selected via the {\tt coolingTimeMethod} parameter.

\subsubsection{Simple}

Selcted with {\tt coolingTimeMethod}$=${\tt simple}, this method assumes that the cooling time is simply
\begin{equation}
 t_{\rm cool} = {N \over 2} {{\rm k}_{\rm B} T n_{\rm tot} \over \Lambda},
\end{equation}
where $N=${\tt coolingTimeSimpleDegreesOfFreedom} is the number of degrees of freedom in the cooling gas which has temperature $T$ and total particle number density (including electrons) $n_{\rm tot}$ and $\Lambda$ is the cooling function.

\subsection{Time Available for Cooling}\label{sec:TimeAvailableCooling}\index{cooling!time available}

The method used to determine the time available for cooling (i.e. the time for which gas in a halo has been able to cool) is selected by the {\tt coolingTimeAvailableMethod} parameter.

\subsubsection{Halo Formation}

Selected with {\tt coolingTimeAvailableMethod}$=${\tt halo formation}, this methods assumes that the time available for cooling is equal to
\begin{equation}
 t_{\rm available} = t - t_{\rm form},
\end{equation}
where $t_{\rm form}$ is the time at which the halo formed (see \S\ref{sec:ComponentFormationTimes}).

\subsubsection{White \& Frenk}

Selected with {\tt coolingTimeAvailableMethod}$=${\tt White-Frenk}, this methods assumes that the time available for cooling is equal to
\begin{equation}
 t_{\rm available} = \exp\left[ f \ln t_{\rm Universe} + (1-f)\ln t_{\rm dynamical} \right],
\end{equation}
where $f=${\tt coolingTimeAvailableAgeFactor} is an interpolating factor, $t_{\rm Universe}$ is the age of the Universe and $t_{\rm dynamical}$ is the dynamical time in the halo. The original \cite{white_galaxy_1991} algorithm corresponds to $f=1$.

\section{Cosmology}

The method used to compute cosmological relations (e.g. expansion factor as a function of time) is selected by the {\tt cosmologyMethod} parameter.

\subsection{Matter + Cosmological Constant Universes}

Selected with {\tt cosmologyMethod}$=${\tt Matter + Lambda}, this method assumes a universe which contains only matter and a cosmological constant

\section{Dark Matter Halos}

Several algorithms are used to implement dark matter halos.

\subsection{Mass Accretion History}\index{mass accretion history!dark matter halo}\index{dark matter halo!mass accretion history}

The method used to compute mass accretion histories of dark matter halos is selected via the {\tt darkMatterAccretionHistoryMethod} parameter.

\subsubsection{Wechsler et al. (2002)}

Selected with {\tt darkMatterAccretionHistoryMethod}$=${\tt Wechsler 2002}, under this method the mass accretion history is given by \citep{wechsler_concentrations_2002}:
\begin{equation}
M(t) = M(t_0) \exp \left( - 2 a_{\rm c} \left[ {a(t_0)\over a(t)}-1 \right] \right),
\end{equation}
where $t_0$ is some reference time and $a_{\rm c}$ is a characteristic expansion factor defined by \cite{wechsler_concentrations_2002} to correspond to the formation time of the halo (using the formation time definition of \citealt{bullock_profiles_2001}).

\subsubsection{Zhao et al. (2009)}

Selected with {\tt darkMatterAccretionHistoryMethod}$=${\tt Zhao 2009}, under this method the algorithm given by \cite{zhao_accurate_2009} to compute mass accretion histories. In particular, \cite{zhao_accurate_2009} give a fitting function for the quantity ${\rm d} \ln \sigma(M)/{\rm d} \ln  \delta_{\rm c}(t)$ for the dimensionless growth rate in a mass accretion history at time $t$ and halo mass $M$. This is converted to a dimensionful growth rate using
\begin{equation}
 {{\rm d} M \over {\rm d} t} = \left({{\rm d} \ln \sigma(M) \over {\rm d} \ln M}\right)^{-1} \left({{\rm d} \delta_c(t) \over {\rm d} t}\right) \left( {M \over \delta_{\rm c}(t)} \right) \left({{\rm d} \ln \sigma(M) \over {\rm d} \ln \delta_{\rm c}(t)}\right).
\end{equation}
This differential equation is then solved numerically to find the mass accretion history.

\subsection{Density Profile}

The method uses to compute density profiles of dark matter halos is selected via the {\tt darkMatterProfileMethod} parameter.

\subsubsection{Isothermal}

Selected with {\tt darkMatterProfileMethod}$=${\tt isothermal}, under this method the density profile is given by:
\begin{equation}
 \rho_{\rm dark matter}(r) \propto r^{-2},
\end{equation}
normalized such that the total mass of the node is enclosed with the virial radius.

\subsubsection{NFW}\index{Navarro-Frenk-White profile}\index{density profile!Navarro-Frenk-White}

Selected with {\tt darkMatterProfileMethod}$=${\tt NFW}, under this method the \NFW\ density profile \citep{navarro_universal_1997} is used
\begin{equation}
  \rho_{\rm dark matter}(r) \propto \left({r\over r_{\rm s}}\right)^{-1} \left[1 + \left({r\over r_{\rm s}}\right) \right]^{-2},
\end{equation}
normalized such that the total mass of the node is enclosed with the virial radius and with the scale length $r_{\rm s} = r_{\rm virial}/c$ where $c$ is the halo concentration (see \S\ref{sec:DarkMatterProfileConcentration}).

\subsubsection{Einasto}\index{Einasto profile}\index{density profile!Einasto}

Selected with {\tt darkMatterProfileMethod}$=${\tt Einasto}, under this method the Einasto density profile (e.g. \citealt{cardone_spherical_2005}) is used
\begin{equation}
  \rho_{\rm dark matter}(r) = \rho_{-2} \exp \left( - {2 \over \alpha} \left[ \left( {r \over r_{-2}} \right)^\alpha - 1 \right] \right),
\end{equation}
normalized such that the total mass of the node is enclosed with the virial radius and with the characteristic length $r_{-2} = r_{\rm virial}/c$ where $c$ is the halo concentration (see \S\ref{sec:DarkMatterProfileConcentration}). The shape parameter, $\alpha$, is set using the density profile shape method (see \S\ref{sec:DarkMatterProfileShape}).

\subsection{Density Profile Concentration}\label{sec:DarkMatterProfileConcentration}\index{dark matter profile!concentration}

The method uses to compute the concentrations of dark matter profiles is selected via the {\tt darkMatterConcentrationMethod} parameter.

\subsubsection{Navarro, Frenk \& White (1996)}

Selected with {\tt darkMatterConcentrationMethod}$=${\tt NFW 1996}, under this method the concentration is computed using the algorithm from \cite{navarro_structure_1996}. In this algoritm, for a given halo of mass $M$ at time $t_0$, a formation time is defined as the epoch at which there is a 50\% probability (according to extended Press-Schechter theory) for a progenitor halo to have a mass greater than $fM$, where $f=${\tt [nfw96ConcentrationF]} is a parameter of the algorithm. This implies formation when the critical overdensity for collapse is
\begin{equation}
 \delta_{\rm crit}(t_{\rm form}) = \left[ 2 \nu_{1/2}^2 \left\{\sigma(fM)^22-\sigma(M)^2\right\} \right]^{1/2}+\delta_{\rm crit}(t_0),
\end{equation}
where $\nu_{1/2} = [\hbox{erfc}^{-1}(1/2)]^{1/2}$. \cite{navarro_structure_1996} then assume an overdensity at collapse of 
\begin{equation}
 \Delta(t_{\rm form}) = C  \left[ {a(t_0) \over a(t_{\rm form})} \right]^3
\end{equation}
where $C=${\tt [nfw96ConcentrationC]} is a parameter of the algorithm. The concentration is then determined by solving
\begin{equation}
 {\Delta(t_{\rm form}) \over \Delta_{\rm virial}(t_0)} = {c^3 \over 3 [\ln(1+c)-c/(1+c)]}.
\end{equation}

\subsubsection{Gao (2008)}

Selected with {\tt darkMatterConcentrationMethod}$=${\tt Gao 2008}, under this method the concentration is computed using a fitting function from \cite{gao_redshift_2008}:
\begin{equation}
\log_{10} c = A \log_{10} M_{\rm halo} + B.
\end{equation}
The parameters are a functon of expansion factor, $a$. We use the following fits to the \cite{gao_redshift_2008} results:
\begin{eqnarray}
A &=& -0.140 \exp\left[-\left(\left\{\log_{10}a+0.05\right\}/0.35\right)^2\right], \\
B &=&  2.646 \exp\left[-\left(\log_{10}a/0.50\right)^2\right].
\end{eqnarray}

\subsubsection{Zhao (2009)}

Selected with {\tt darkMatterConcentrationMethod}$=${\tt Zhao 2009}, under this method the concentration is computed using a fitting function from \cite{zhao_accurate_2009}:
\begin{equation}
 c = 4 \left(1 + \left[ {t  \over 3.75 t_{\rm form}}\right]^{8.4}\right)^{1/8},
\end{equation}
where $t$ is the time for the halo and $t_{\rm form}$ is a formation time defined by \cite{zhao_accurate_2009} as the time at which the main branch progenitor of the halo had a mass equal to $0.04$ of the current halo mass. This formation time is computed directly from the merger tree branch associated with each halo. If the no branch exists or does not extend to the formation time then the formation time is computed by extrapolating the mass of the earliest resolved main branch progenitor to earlier times using the selected mass accretion history method (see \S\ref{sec:HaloMassAccretionHistory}).

\subsubsection{Mu\~noz-Cuartas (2011)}

Selected with {\tt darkMatterConcentrationMethod}$=${\tt Munoz-Cuartas 2011}, under this method the concentration is computed using a fitting function from \cite{munoz-cuartas_redshift_2011}:
\begin{equation}
\log_{10} c = a \log_{10} \left( {M_{\rm halo} \over h^{-1}M_\odot} \right) + b.
\end{equation}
The parameters are a functon of redshift, $z$, given by
\begin{eqnarray}
a &=& wz-m, \\
b &=& {\alpha \over (z+\gamma)} + {\beta \over (z+\gamma)^2},
\end{eqnarray}
where $w=0.029$, $m=0.097$, $\alpha=-110.001$, $\beta=2469.720$, $\gamma=16.885$.

\subsection{Density Profile Shape}\label{sec:DarkMatterProfileShape}\index{dark matter profile!shape}

The method used to compute any shape parameter of dark matter profiles is selected via the {\tt darkMatterShapeMethod} parameter.

\subsubsection{Gao (2008)}

Selected with {\tt darkMatterShapeMethod}$=${\tt Gao 2008}, under this method the shape parameter for Einasto density profiles\index{Einasto profile}\index{density profile!Einasto} is computed using a fitting function from \cite{gao_redshift_2008}:
\begin{equation}
\alpha = \left\{ \begin{array}{ll} 0.155 + 0.0095\nu^2 & \hbox{ if } \nu < 3.907 \\ 0.3 & \hbox{ if } \nu \ge 3.907, \end{array} \right.
\end{equation}
where $\nu=\delta_{\rm c}(t)/\sigma(M)$ is the peak height of the halo. The truncation at $\alpha = 0.3$ is included since \cite{gao_redshift_2008}'s fits do not probe this region and extremely large values of $\alpha$ are numerically troublesome.

\subsection{Mass Loss Rates}\index{dark matter halo!mass loss}\index{mass loss!dark matter halo}

The method used to compute the rate of mass loss from dark matter (sub)halos is selected via the {\tt darkMatterHaloMassLossRateMethod} parameter.

\subsubsection{Null}

Selected with {\tt darkMatterHaloMassLossRateMethod}$=${\tt null}, this method assumes a zero rate of mass loss from dark matter halos.

\subsubsection{van den Bosch et al. (2005)}

Selected with {\tt darkMatterHaloMassLossRateMethod}$=${\tt vanDenBosch2005}, this method uses the algorithm of \cite{van_den_bosch_mass_2005} to compute the rate of mass loss. Specifically:
\begin{equation}
\dot{M}_{\rm node,bound} = -{M_{\rm node,bound}\over \tau} \left({M_{\rm node,bound} / M_{\rm node,parent}}\right)^\zeta,
\end{equation}
where $M_{\rm node,parent}$ is the mass of the parent node in which the halo lives and
\begin{equation}
\tau = \tau_0 \left({\Delta_{\rm vir}(t) \over \Delta(t_0)}\right)^{-1/2} a^{3/2},
\end{equation}
where $\Delta_{\rm vir}(t)$ is the virial overdensity of halos at time $t$ and $a$ is the expansion factor. The fitting parameters, $\tau_0$ and $\zeta$ have values of 0.13~Gyr and 0.36 respectively as determined by \cite{van_den_bosch_mass_2005}. Note that  \cite{van_den_bosch_mass_2005} write this expression in a slightly different form since their $\Delta_{\rm vir}$ is defined relative to the critical density rather than the mean density as it is in \glc. In both cases, the timescale $\tau$ simply scales as $\langle \rho_{\rm vir} \rangle ^{-1/2}$ where $\langle \rho_{\rm vir} \rangle$ is the mean virial overdensity of halos.

\subsection{Spin Parameter Distribution}\label{sec:SpinParameterDistribution}\index{dark matter halo!spin!distribution}\index{spin!dark matter halo}

The method used to compute the distribution of dark matter halo spin parameters is selected via the {\tt haloSpinDistributionMethod} parameter.

\subsubsection{Lognormal}

Selected with {\tt haloSpinDistributionMethod}$=${\tt lognormal}, under this method the spin is drawn from a lognormal distribution with median {\tt [lognormalSpinDistributionMedian]} and width {\tt [lognormalSpinDistributionSigma]}.

\subsubsection{Bett et al. (2007)}

Selected with {\tt haloSpinDistributionMethod}$=${\tt Bett2007}, under this method the spin is drawn from the distribution found by \cite{bett_spin_2007}. The $\lambda_0$ and $\alpha$ parameter of Bett et al.'s distribution are set by the {\tt [spinDistributionBett2007Lambda0]} and {\tt [spinDistributionBett2007Alpha]} input parameters.

\subsubsection{Delta Function}

Selected with {\tt haloSpinDistributionMethod}$=${\tt delta function}, under this method the spin is drawn from a delta function distribution, $P(\lambda) = \delta(\lambda-\lambda_0)$, where $\lambda_0=${\tt [deltaFunctionSpinDistributionSpin]}, i.e. a fixed value of spin equal to $\lambda_0$ is returned.

\section{Disk Stability/Bar Formation}\label{sec:DiskStability}\index{disks!stability}\index{bar instability}

The method uses to compute the bar instability timescale for galactic disks is selected via the {\tt barInstabilityMethod} parameter.

\subsection{Efstathiou, Lake \& Negroponte}

Selected with {\tt barInstabilityMethod}$=${\tt ELN}, this method uses the stability criterion of \cite{efstathiou_stability_1982} to estimate when disks are unstable to bar formation:
\begin{equation}
 \epsilon \left( \equiv {V_{\rm peak} \over \sqrt{\G M_{\rm disk}/r_{\rm disk}}} \right) < \epsilon_{\rm c},
\end{equation}
for stability, where $V_{\rm peak}$ is the peak velocity in the rotation curve (computed here assuming an isolated exponential disk), $M_{\rm disk}$ is the mass of the disk and $r_{\rm disk}$ is its scale length (assuming an exponential disk). The value of $\epsilon_{\rm c}$ is linearly interpolated in the disk gas fraction between values for purely gaseous and stellar disks as specified by {\tt stabilityThresholdStellar} and {\tt stabilityThresholdGaseous} respectively. For disks which are judged to be unstable, the timescale for bar formation is estimated to be
\begin{equation}
 t_{\rm bar} = t_{\rm disk} {\epsilon_{\rm c} - \epsilon_{\rm iso} \over \epsilon_{\rm c} - \epsilon},
\end{equation}
where $\epsilon_{\rm iso}$ is the value of $\epsilon$ for an isolated disk and $t_{\rm disk}$ is the disk dynamical time, defined as $r/V$, at one scale length. This form gives an infinite timescale at the stability threshold, reducing to a dynamical time for highly unstable disks.

\section{Galactic Structure}\index{galactic structure}

The algorithm to be used when solving for galactic structure (specifically, finding radii of galactic components) is selected via the {\tt galacticStructureRadiusSolverMethod} parameter.

\subsection{Simple}

Selected with {\tt galacticStructureRadiusSolverMethod}$=${\tt simple} this method determines the sizes of galactic components by assuming that their self-gravity is negligible (i.e. that the gravitational potential well is dominated by dark matter) and that, therefore, baryons do not modify the dark matter density profile. The radius of a given component is then found by solving
\begin{equation}
 j = \sqrt{\G M_{\rm DM}(r) r},
\end{equation}
where $j$ is the specific angular momentum of the component (at whatever point in the profile is to be solved for), $r$ is radius and $M(r)$ is the mass of dark matter within radius $r$.

\subsection{Adiabatic}

Selected with {\tt galacticStructureRadiusSolverMethod}$=${\tt adiabatic}, this method takes into account the baryonic self-gravity of all galactic components when solving for structure and additionally accounts for backreaction of the baryons on the dark matter density profile using the adiabatic contraction algorithm of \cite{gnedin_response_2004}. The parameter $A$ and $\omega$ of that model are specified via input parameters {\tt adiabaticContractionGnedinA} and {\tt adiabaticContractionGnedinOmega} respectively. Solution proceeds via an iterative procedure to find equilibrium radii for all galaxies in a consistently contracted halo. The method used follows that described by \cite{benson_galaxy_2010}.

\section{Galaxy Merging}\index{galaxy!merging}\index{merging!galaxy}

The process of merging two galaxies currently involves two algorithms: one which decides how the merger causes mass components from both galaxies to move and one which determines the size of the remnant galaxy spheroid.

\subsection{Mass Movements}\label{sec:MergingMassMovements}

The movement of mass elements in the merging galaxies is determined by the {\tt satelliteMergingMassMovementsMethod} parameter. 

\subsubsection{Simple}

Selected with {\tt satelliteMergingMassMovementsMethod}$=${\tt simple}, this method implements mass movements according to:
\begin{itemize}
 \item If $M_{\rm satellite} > f_{\rm major} M_{\rm central}$ then all mass from both satellite and central galaxies moves to the spheroid component of the central galaxy;
 \item Otherwise: Gas from the satellite moves to the component of the central specified by the {\tt [minorMergerGasMovesTo]} parameter (either ``{\tt disk}'' or ``{\tt spheroid}''), stars from the satellite moves to the spheroid of the central and mass in the central does not move.
\end{itemize}
Here, $f_{\rm major}=${\tt [majorMergerMassRatio]} is the mass ratio above which a merger is considered to be ``major''.

\subsubsection{Baugh et al. (2005)}

Selected with {\tt satelliteMergingMassMovementsMethod}$=${\tt Baugh2005}, this method implements mass movements according to:
\begin{itemize}
 \item If $M_{\rm satellite} > f_{\rm major} M_{\rm central}$ then all mass from both satellite and central galaxies moves to the spheroid component of the central galaxy;
 \item Otherwise:
 \begin{itemize}
  \item If $M_{\rm satellite} > f_{\rm burst} M_{\rm central}$ and the gas fraction in the host exceeds $f_{\rm gas,crit}$ then all gas is moved to the host spheroid, while the host stellar disk remains in place.
  \item Otherwise, gas from the satellite moves to the component of the central specified by the {\tt [minorMergerGasMovesTo]} parameter (either ``{\tt disk}'' or ``{\tt spheroid}''), stars from the satellite moves to the spheroid of the central and mass in the central does not move.
 \end{itemize}
\end{itemize}
Here, $f_{\rm major}=${\tt [majorMergerMassRatio]} is the mass ratio above which a merger is considered to be ``major'', while $f_{\rm burst}=${\tt [burstMassRatio]} and $f_{\rm gas,crit}=${\tt [burstCriticalGasFraction]}.

\subsection{Remnant Sizes}\index{merging!remnant size}

The method used to calculate the sizes of merger remnant spheroids is selected by the {\tt satelliteMergingRemnantSizeMethod} parameter.

\subsubsection{Null}

Selected using {\tt satelliteMergingRemnantSizeMethod}$=${\tt null}, this is a null method which does nothing at all. It is useful, for example, when running \glc\ to study dark matter only (i.e. when no galaxy properties are computed).

\subsubsection{Cole et al. (2000)}\label{sec:MergerRemnantSizeCole2000}

Selected using {\tt satelliteMergingRemnantSizeMethod}$=${\tt Cole2000}, this method uses the algorithm of \cite{cole_hierarchical_2000} to compute merger remnant spheroid sizes. Specifically
\begin{equation}
\frac{(M_1+M_2)^2}{ r_{\rm new}} =
\frac{M_1^2}{r_1} + \frac{M_2^2}{r_2} + \frac{ f_{\rm orbit}}{c}
\frac{M_1 M_2}{r_1+r_2},
\end{equation}
where $M_1$ and $M_2$ are the baryonic masses of the components of the merging galaxies that will end up in the spheroid component of the remnant\footnote{Depending on the merging rules (see \S\protect\ref{sec:MergingMassMovements}) not all mass may be placed into the spheroid component of the remnant.} and $r_1$ and $r_2$ are the half mass radii of those same components of the merging galaxies\footnote{In practice, \glc\ computes a weighted average of the disk and spheroid half-mass radii of each galaxy, with weights equal to the masses of each component (disk and spheroid) which will become part of the spheroid component of the remnant.}, $r_{\rm new}$ is the half mass radius of the spheroidal component of the remnant galaxy and $c$ is a constant which depends on the distribution of the mass. For a Hernquist spheroid $c=0.40$ can be found by numerical integration while for a exponential disk $c=0.49$. For simplicity a value of $c=0.5$ is adopted for all components. The parameter $f_{\rm orbit}=${\tt mergerRemnantSizeOrbitalEnergy} depends on the orbital parameters of the galaxy pair. For example, a value of $f_{\rm orbit} = 1$ corresponds to point mass galaxies in circular orbits about their center of mass. 

A subtelty arises because the above expression accounts for only the baryonic mass of material which becomes part of the spheroid component of the remnant. In reality, there are additional terms in the energy equation due to the interaction of this material with any dark matter mass in each galaxy and any baryonic mass of each galaxy which does not become part of the spheroid component of the remnant. To account for this additional matter, an effective boost factor, $f_{\rm boost}$, to the specific angular momentum of each component of each merging galaxy is computed:
\begin{equation}
 f_{\rm boost} = {j \over \sqrt{{\rm G} M r_{1/2}}},
\end{equation}
where $j$ is the specific angular momentum of the component, $M$ is its total baryonic mass and $r_{\rm 1/2}$ is its half-mass radius. The mass-weighted mean boost factor is found by combining those of all components which will form part of the spheroid of the remnant. The final specific angular momentum of the remnant spheroid is then given by:
\begin{equation}
 j_{\rm new} = \langle f_{\rm boost} \rangle r_{\rm new} V_{\rm new},
\end{equation}
where
\begin{equation}
 V_{\rm new}^2 = {{\rm G} (M_1+M_2)\over r_{\rm new}}.
\end{equation}

\subsubsection{Covington et al. (2008)}

Selected using {\tt satelliteMergingRemnantSizeMethod}$=${\tt Covington2008}, this method uses the algorithm of \cite{covington_predicting_2008} to compute merger remnant spheroid sizes. Specifically
\begin{equation}
\frac{(M_1+M_2)^2}{ r_{\rm new}} =
\left[ \frac{M_1^2}{r_1} + \frac{M_2^2}{r_2} + \frac{ f_{\rm orbit}}{c}
\frac{M_1 M_2}{r_1+r_2}\right] \left( 1 + f_{\rm gas} C_{\rm rad} \right),
\label{eq:Covington2008Radius}
\end{equation}
where $M_1$ and $M_2$ are the baryonic masses of the merging galaxies and $r_1$
and $r_2$ are their half mass radii, $r_{\rm new}$ is the half mass radius of the spheroidal component of the remnant galaxy and $c$ is a constant which depends on the distribution of the mass. For a Hernquist spheroid $c=0.40$ can be found by numerical integration while for a exponential disk $c=0.49$. For simplicity a value of $c=0.5$ is adopted for all components. The parameter $f_{\rm orbit}=${\tt mergerRemnantSizeOrbitalEnergy} depends on the orbital parameters of the galaxy pair. For example, a value of $f_{\rm orbit} = 1$ corresponds to point mass galaxies in circular orbits about their center of mass. The final term on the right hand side of eqn.~(\ref{eq:Covington2008Radius}) gives a correction to the final energy of the remnant due to dissipational losses based on the results of \cite{covington_effects_2011}, with
\begin{equation}
 f_{\rm gas} = {M_{\rm 1,gas}+M_{\rm 2,gas} \over M_1+M_2}
\end{equation}
begin the gas fraction of the progenitor galaxies. By default, $C_{\rm rad}=2.75$ \citep{covington_effects_2011}. To account for the effects of dark matter and non-spheroid baryonic matter the same approach is used as in the \cite{cole_hierarchical_2000} algorithm (see \S\ref{sec:MergerRemnantSizeCole2000}). 

\section{Hot Halo Density Profile}\index{hot halo!density profile}\index{density profile!hot halo}

The algorithm to be used when determining the hot halo density profile is selected via the {\tt hotHaloDensityMethod} parameter.

\subsection{Cored Isothermal}

Selected with {\tt hotHaloDensityMethod}$=${\tt cored isothermal} this method adopts a spherically symmetric cored-isothermal density profile for the hot halo. Specifically,
\begin{equation}
 \rho_{\rm hot halo}(r) \propto \left[ r^2 + r_{\rm core}^2 \right]^2,
\end{equation}
where the core radius, $r_{\rm core}$, is set using the selected cored isothermal core radius method (see \S\ref{sec:hotHaloDensityProfileCoredIsothermalCoreRadius}). The profile is normalized such that the current mass in the hot gas profile is contained within the virial radius.

\subsection{Null}

Selected with {\tt hotHaloDensityMethod}$=${\tt null} this method assumes no hot halo density profile. It is useful, for example, when performing dark matter-only calculations.

\section{Hot Halo Density Profile: Cored Isothermal Core Radius}\label{sec:hotHaloDensityProfileCoredIsothermalCoreRadius}\index{hot halo!density profile!core radius}\index{density profile!hot halo!core radius}

The algorithm to be used when determining the core radius of cored isothermal hot halo density profiles is selected via the {\tt hotHaloCoredIsothermalCoreRadiiMethod} parameter.

\subsection{Growing Core}

Selected with {\tt hotHaloCoredIsothermalCoreRadiiMethod}$=${\tt growing core}, this method implements a core radius equal to a fraction {\tt [isothermalCoreRadiusOverScaleRadius]} of the node's dark matter profile scale radius for nodes containing a mass of hot gas equal to the universal baryon fraction times their total mass. For nodes containing less hot gas mass, the core radius is expanded to maintain the same gas density at the virial radius, with a maximum core radius of {\tt [isothermalCoreRadiusOverVirialRadiusMaximum]} times the node's virial radius.

\subsection{Virial Radius Fraction}

Selected with {\tt hotHaloCoredIsothermalCoreRadiiMethod}$=${\tt virialRadiusFraction}, this method implements a core radius equal to a fraction {\tt [isothermalCoreRadiusOverVirialRadius]} of the node's virial radius.

\section{Hot Halo Temperature Profile}\label{sec:HotHaloTemperature}\index{hot halo!temperature profile}\index{temperature profile!hot halo}

The algorithm to be used when determining the hot halo temperature profile is selected via the {\tt hotHaloTemperatureMethod} parameter.

\subsection{Virial Temperature}

Selected with {\tt hotHaloTemperatureMethod}$=${\tt virial} this method assumes an isothermal halo with a temperature equal to the virial temperature of the halo.

\section{Initial Mass Functions}\index{initial mass function}

The stellar \IMF\ subsystem supports multiple IMFs and extensible algorithms to select which \IMF\ to use based on the physical conditions of star formation.

\subsection{Initial Mass Function Selection}\index{initial mass function!selection}

The method to use for selecting which \IMF\ to use is specified by the {\tt imfSelectionMethod} parameter.

\subsubsection{Fixed}

Selected by {\tt imfSelectionMethod}$=${\tt fixed}, this method uses a fixed \IMF\ irrespective of physical conditions. The \IMF\ to use is specified by the {\tt imfSelectionFixed} parameter (e.g. setting this parameter to {\tt Salpeter} selects the Salpeter \IMF).

\subsubsection{Disk and Spheroid}

Selected by {\tt imfSelectionMethod}$=${\tt diskSpheroid}, this method uses different {\IMF}s for star formation in disks and spheroids irrespective of other physical conditions. The {\IMF}s to use are specified by the {\tt imfSelectionDisk} and {\tt imfSelectionSpheroid} parameters (e.g. setting one of these parameters to {\tt Salpeter} selects the Salpeter \IMF).

\subsection{Initial Mass Functions}\label{sec:physicsIMF}\index{initial mass function}

A variety of different \IMF s are available. Each \IMF\ registers itself with the \glc\ \IMF\ subsystem and can then be looked-up by name or internal index. All IMFs are assumed to be continuous in $M$, unless otherwise noted and normalized to unit mass. Each \IMF\ supplies a recycled fraction and metal yield for use in the instantaneous recycling approximation. These can be set via the parameters {\tt imf[imfName]RecycledInstantaneous} and {\tt imf[imfName]YieldInstantaneous} where {\tt [imfName]} is the name of the \IMF. Their default values were computed using \glc 's internal stellar astrophysics modules for a Solar metallicity population with age of $13.8$ Gyr.

\subsubsection{Baugh et al. (2005) Top-Heavy}

The {\tt Baugh2005TopHeavy} \IMF\ is defined by \citep{baugh_can_2005}:
\begin{equation}
 \phi(M) \propto 
 M^{-1} \hbox{ for } 0.15M_\odot < M < 125M_\odot
\end{equation}

\subsubsection{Chabrier}

The {\tt Chabrier} \IMF\ is defined by \citep{chabrier_galactic_2001}:
\begin{equation}
 \phi(M) \propto \left\{ \begin{array}{ll}
 M^{-1} \exp(-[\log_{10}(M/M_{\rm c})/\sigma_{\rm c}]^2/2) & \hbox{ for } 0.1M_\odot < M < 1M_\odot \\
 M^{-2.3} & \hbox{ for } 1M_\odot < M < 125M_\odot \\
 0 & \hbox {otherwise,} \end{array} \right.
\end{equation}
where $\sigma_{\rm c}=0.69$ and $M_{\rm c}=0.08M_\odot$.

\subsubsection{Kennicutt}

The {\tt Kennicutt} \IMF\ is defined by \citep{kennicutt_rate_1983}:
\begin{equation}
 \phi(M) \propto \left\{ \begin{array}{ll}
 M^{-1.25} & \hbox{ for } 0.10M_\odot < M < 1.00M_\odot \\
 M^{-2.00} & \hbox{ for } 1.00M_\odot < M < 2.00M_\odot \\
 M^{-2.30} & \hbox{ for } 2.00M_\odot < M < 125M_\odot \\
 0 & \hbox {otherwise.} \end{array} \right.
\end{equation}

\subsubsection{Kroupa}

The {\tt Kroupa} \IMF\ is defined by \citep{kroupa_variation_2001}:
\begin{equation}
 \phi(M) \propto \left\{ \begin{array}{ll}
 M^{-0.3} & \hbox{ for } 0.01M_\odot < M < 0.08M_\odot \\ 
 M^{-1.8} & \hbox{ for } 0.08M_\odot < M < 0.5M_\odot \\ 
 M^{-2.7} & \hbox{ for } 0.5M_\odot < M < 1M_\odot \\ 
 M^{-2.3} & \hbox{ for } 1M_\odot < M < 125M_\odot \\ 
0 & \hbox {otherwise.} \end{array} \right.
\end{equation}

\subsubsection{Miller-Scalo}

The {\tt Miller-Scalo} \IMF\ is defined by \citep{miller_initial_1979}:
\begin{equation}
 \phi(M) \propto \left\{ \begin{array}{ll}
 M^{-1.25} & \hbox{ for } 0.10M_\odot < M < 1.00M_\odot \\
 M^{-2.00} & \hbox{ for } 1.00M_\odot < M < 2.00M_\odot \\
 M^{-2.30} & \hbox{ for } 2.00M_\odot < M < 10.0M_\odot \\
 M^{-3.30} & \hbox{ for } 10.0M_\odot < M < 125M_\odot \\
 0 & \hbox {otherwise.} \end{array} \right.
\end{equation}

\subsubsection{Salpeter}

The {\tt Salpeter} \IMF\ is defined by \citep{salpeter_luminosity_1955}:
\begin{equation}
 \phi(M) \propto \left\{ \begin{array}{ll} M^{-2.35} & \hbox{ for } 0.1M_\odot < M < 125M_\odot \\ 0 & \hbox {otherwise.} \end{array} \right.
\end{equation}

\subsubsection{Scalo}

The {\tt Scalo} \IMF\ is defined by \citep{scalo_stellar_1986}:
\begin{equation}
 \phi(M) \propto \left\{ \begin{array}{ll}
 M^{+1.60} & \hbox{ for } 0.10M_\odot < M < 0.18M_\odot \\
 M^{-1.01} & \hbox{ for } 0.18M_\odot < M < 0.42M_\odot \\
 M^{-2.75} & \hbox{ for } 0.42M_\odot < M < 0.62M_\odot \\
 M^{-2.08} & \hbox{ for } 0.62M_\odot < M < 1.18M_\odot \\
 M^{-3.50} & \hbox{ for } 1.18M_\odot < M < 3.50M_\odot \\
 M^{-2.63} & \hbox{ for } 3.50M_\odot < M < 125M_\odot \\
 0 & \hbox {otherwise.} \end{array} \right.
\end{equation}







\section{Intergalactic Medium State}\label{sec:IntergalacticMediumStateMethod}\index{intergalactic medium}

The thermal and ionization state of the intergalactic medium is implemented by the algorithm selected using the {\tt intergalaticMediumStateMethod} parameter.

\subsection{{\sc RecFast}}

Selected using {\tt intergalaticMediumStateMethod}$=${\tt RecFast}, this method computes the state of the intergalactic medium using the \href{http://www.astro.ubc.ca/people/scott/recfast.html}{{\sc RecFast}} code \cite{seager_how_2000,wong_how_2008}. The {\sc RecFast} code will be downloaded and run to compute the intergalactic medium state as needed, which will then be stored for future use.

\subsection{File}

Selected using {\tt intergalaticMediumStateMethod}$=${\tt file}, this method reads the state of the intergalactic medium from a file and interpolates in the tabulated results. The format of the file is specified in \S\ref{sec:IntergalacticMediumStateMethods}.

\section{Ionization State}\label{sec:IonizationStateMethod}\index{ionization state}

The ionization state of gas is implemented by the algorithm selected using the {\tt ionizatonStateMethod} parameter.

\subsection{Atomic Collisional Ionization Equilibrium Using {\sc Cloudy}}

Selected using {\tt ionizatonStateMethod}$=${\tt atomic\_CIE\_Cloudy}, this method computes the ionization state using the {\sc Cloudy} code and under the assumption of collisional ionization equilibrium with no molecular contribution. Abundances are Solar, except for zero metallicity calculations which use {\sc Cloudy}'s ``primordial'' metallicity. The helium abundance for non-zero metallicity is scaled between primordial and Solar values linearly with metallicity. The {\sc Cloudy} code will be downloaded and run to compute the cooling function as needed, which will then be stored for future use. As this process is slow, a precomputed table is provided with \glc. If metallicities outside the range tabulated in this file are required it will be regenerated with an appropriate range.

\subsection{Collisional Ionization Eqiulibruim From File}

Selected using {\tt ionizatonStateMethod}$=${\tt CIE\_from\_file}, in this method the ionization state is read from a file specified by the {\tt ionizationStateFile} parameter. The format of this file is specified in \S\ref{sec:IonizationStateMethods}. The ionization state is assumed to be computed under conditions of collisional ionization equilibrium and therefore densities scale as $\rho$. Optional H{\sc i} and H{\sc ii} densities, if present in the file, will be read and used when returning the densities of ``molecular'' species.

\section{Merger Tree Construction}\index{merger trees}

Merger trees are ``constructed\footnote{By ``construct'' we mean any process of creating a representation of a merger tree within \protect\glc.}'' by the method specified by the {\tt mergerTreeConstructMethod} parameter.

\subsection{Read From File}

Selected with {\tt mergerTreeConstructMethod}$=${\tt read}, this method reads merger tree structures from an HDF5 file specified by the {\tt mergerTreeReadFileName} parameter. The structure of these HDF5 files is described in \S\ref{sec:MergerTreeFiles}.

\subsection{Build}

Selected with {\tt mergerTreeConstructMethod}$=${\tt build}, this method first creates a distribution of tree root halo masses and then builds a merger tree using the algorithm specified by the {\tt mergerTreeBuildMethod} parameter.

If {\tt [mergerTreeBuildTreesHaloMassDistribution]}$=${\tt read}, then these masses will be read from a file specified by {\tt [mergerTreeBuildTreeMassesFile]}. Otherwise, the root halo masses are selected to range between {\tt mergerTreeBuildHaloMassMinimum} and {\tt mergerTreeBuildHaloMassMaximum} with {\tt mergerTreeBuildTreesPerDecade} trees per decade of root halo mass on average. Trees are rooted at {\tt mergerTreeBuildTreesBaseRedshift} and tree building will begin with the {\tt mergerTreeBuildTreesBeginAtTree}$^{\rm th}$ tree\footnote{This will normally be set to 1 to begin with the first tree. Other values allow to begin on later trees for debugging purposes.}. The distribution of halo masses is such that the mass of the $i^{\rm th}$ halo is
\begin{equation}
 M_{\rm halo,i} = \exp\left[ \ln(M_{\rm halo,min}) + \ln\left({M_{\rm halo,max}/M_{\rm halo,min}}\right) x_i^{1+\alpha} \right].
\end{equation}
Here, $x_i$ is a number between 0 and 1 and $\alpha=${\tt mergerTreeBuildTreesHaloMassExponent} is an input parameter that controls the relative number of low and high mass tree produced. The distribution of $x$ is determined by the input parameter {\tt mergerTreeBuildTreesHaloMassDistribution} with options:
\begin{description}
 \item [{\tt uniform}] $x$ is distributed uniformly between 0 and 1;
 \item [{\tt quasi}] $x$ is distributed using a quasi-random sequence.
\end{description}

In the case of reading root halo masses from a file, the file should be an XML file with the following form:
\begin{verbatim}
 <mergerTrees>
  <treeRootMass>13522377303.5998</treeRootMass>
  <treeRootMass>19579530191.8709</treeRootMass>
  <treeRootMass>21061025282.9613</treeRootMass>
  .
  .
  .
 </mergerTrees>
\end{verbatim}
where each {\tt treeRootMass} element gives the mass (in Solar masses) of the root halo of a tree to generate.

\section{Merger Tree Branching}\index{merger trees!branching}

The method to be used for computing branching probabilities in merger trees is specified by the {\tt treeBranchingMethod} parameter.

\subsection{Modified Press-Schechter}

Selected with {\tt treeBranchingMethod}$=${\tt modified Press-Schechter}, this method uses the algorithm of \cite{parkinson_generating_2008} to compute branching ratios. The parameters $G_0$, $\gamma_1$ and $\gamma_2$ of their algorithm are specified by the input parameters {\tt modifiedPressSchechterG0}, {\tt modifiedPressSchechterGamma1} and {\tt modifiedPressSchechterGamma2} respectively. Additionally, the parameter {\tt modifiedPressSchechterFirstOrderAccuracy} limits the step in $\delta_{\rm crit}$ so that it never exceeds {\tt modifiedPressSchechterFirstOrderAccuracy}$\sqrt{2[\sigma^2(M_2/2)-\sigma^2(M_2)]}$, which ensures the the first order expansion of the merging rate that is assumed is accurate.

\section{Merger Tree Building}\index{merger trees!building}

The method to be used for building merger trees is specified by the {\tt mergerTreeBuildMethod} parameter.

\subsection{Cole et al. (2000) Algorithm}

Selected with {\tt mergerTreeBuildMethod}$=${\tt Cole2000}, this method uses the algorithm described by \cite{cole_hierarchical_2000}, with a branching probability method selected via the {\tt treeBranchingMethod} parameter. This action of this algorithm is controlled by the following parameters:
\begin{description}
 \item [{\tt mergerTreeBuildCole2000MergeProbability}] The maximum probability for a binary merger allowed in a single timestep. This allows the probability to be kept small, such the the probability for multiple mergers within a single timestep is small.
 \item [{\tt mergerTreeBuildCole2000AccretionLimit}] The maximum fractional change in mass due to sub-esolution accretion allowed in any given timestep when building the tree.
 \item [{\tt mergerTreeBuildCole2000MassResolution}] The minimum halo mass (in $M_\odot$) that the algorithm will follow. Mass accretion below this scale is treated as smooth accretion and branches are truncated once they fall below this mass.
\end{description}

\subsection{Smooth Accretion}\label{sec:SmoothAccretion}

Selected with {\tt mergerTreeConstructMethod}$=${\tt smoothAccretion}, this method builds a branchless merger tree with a smooth accretion history using the selected mass accretion history method (see \S\ref{sec:HaloMassAccretionHistory}). The tree has a final mass of {\tt mergerTreeHaloMass} (in units of $M_\odot$) at redshift {\tt mergerTreeBaseRedshift} and is continued back in time by decreasing the halo mass by a factor {\tt mergerTreeHaloMassDeclineFactor} at each new node until a specified {\tt mergerTreeHaloMassResolution} (in units of $M_\odot$) is reached.

\section{Merger Tree Pre-evolution Processing}

Arbitrary processing of merger trees prior to their evolution can be carried out using the {\tt mergerTreePreEvolveTask} directive (see \S\ref{sec:MergerTreePreEvolveTask}). Currently defined tasks are defined below.

\subsection{Interpolate Tree to Time Grid}\index{merger tree!fixed time steps}\index{merger tree!regrid times}

This task will interpolate the merger tree structure onto a new array of timesteps if {\tt [mergerTreeRegridTimes]}$=${\tt true}. The timestep array is specified via the parameters:
\begin{description}
\item[{\tt [mergerTreeRegridStartExpansionFactor]}] The smallest expansion factor in the array;
\item[{\tt [mergerTreeRegridEndExpansionFactor]}] The largest expansion factor in the array;
\item[{\tt [mergerTreeRegridCount]}] The number of timesteps in the array;
\item[{\tt [mergerTreeRegridSpacing]}] The spacing of the timesteps. Two options are available: {\tt linear} will space timesteps uniformly in expansion factor, while {\tt logarithmic} will space timesteps uniformly in the logarithm of expansion factor.
\end{description}
Along each branch of the tree, new halos are inserted at times corresponding to the times in the resulting array. The masses of these nodes are linearly interpolated between the existing nodes on the branch. Once these new nodes have been added, all other nodes are removed from the tree\footnote{The base node of the tree is never removed, even if it does not lie on one of the times in the constructed array.} The processing is useful to construct representations of trees as they would be if only sparse time sampling were available. As such, it is useful for exploring how the number of snapshots in merger trees extracted from N-body simulations\index{merger tree!N-body} affects the properties of galaxies that form in them.

\subsection{Mass Accretion History Output}\index{merger tree!mass accretion history}

Output of the mass accretion history (i.e. the mass of the node on the primary branch as a function of time) for each merger tree can be requested by setting {\tt [massAccretionHistoryOutput]}$=${\tt true}. If requested, an additional group, {\tt massAccretionHistories}, is made in the \glc\ output file. This group will contain a subgroup for each merger tree ({\tt mergerTreeN} where {\tt N} is the merger tree index) within which three datasets, {\tt nodeIndex}, {\tt nodeTime} and {\tt nodeMass}, can be found. These give the index, time and mass of the node on the primary branch of the tree at all times for which the tree is defined.

\subsection{Tree Pruning By Mass}\index{merger tree!pruning!by mass}

This task allows for branches of merger trees to be pruned---i.e. nodes below a specified mass limit are removed from the tree prior to any evolution. This can be useful for convergence studies for example. To prune branches set {\tt [mergerTreePruneBranches]}$=${\tt true} and set {\tt [mergerTreePruningMassThreshold]} to the desired mass threshold below which nodes will be pruned.

\subsection{Tree Pruning By Hierarchy}\index{merger tree!pruning!by hierarchy}

This task allows for branches of merger trees to be pruned by hierarchy---i.e. nodes below a given depth in the hierarchy\footnote{The main branch is defined as depth 0. Other branches are assigned a depth equal to the depth of the branch onto which they merge plus 1. For example, any branch which merges directly onto the main branch is defined as depth 1.} are removed from the tree prior to any evolution.  To prune branches by hierarchy depth set {\tt [mergerTreePruneHierarchyAtDepth]} to the desired depth at which to prune. For example, a value of 1 will result in all branches except for the main branch being removed, while a value of 2 will remove all branches that do not merge directly onto the main branch. (Note that setting {\tt [mergerTreePruneHierarchyAtDepth]} to zero will result in no pruning.)

\section{Molecular Reaction Rates}\index{molecules!reaction rates}\index{reaction rates!molecular}\label{sec:MolecularReactionRates}

Methods for computing molecular reaction rates are selected via the {\tt [molecularReactionRatesMethods]} parameter. Multiple methods can be selected---their rates are cumulated.

\subsection{Null}

Selected with {\tt molecularReactionRatesMethods}$=${\tt null} this is a null method which does not set any rates.

\subsection{Hydrogen Network}\index{hydrogen!molecular}

Selected with {\tt molecularReactionRatesMethods}$=${\tt hydrogenNetwork} this method computes rates using the network of reactions and fitting functions from \cite{abel_modeling_1997} and \cite{tegmark_small_1997}. The parameter {\tt [hydrogenNetworkFast]} controls the approximations made. If set {\tt true} then H$^-$ is assumed to be at equilibrium abundance, H$_2^+$ reactions are ignored and other slow reactions are ignored (see \citealt{abel_modeling_1997}).

\section{Star Formation Timescales}\index{star formation!timescale}

The methods for computing star formation timescales in disks and spheroids are selected via the {\tt starFormationTimescaleDisksMethod} and {\tt starFormationTimescaleSpheroidsMethod} respectively.

\subsection{Dynamical Time}

Selected with {\tt starFormationTimescale[Disks|Spheroids]Method}$=${\tt dynamical time} this method computes the star formation timescale to be:
\begin{equation}
 \tau_\star = \epsilon_\star^{-1} \tau_{\rm dynamical} \left( {V \over 200\hbox{km/s}} \right)^{\alpha_\star},
\end{equation}
where $\epsilon_\star=${\tt starFormation[Disks|Spheroids]Efficiency} and $\alpha_\star=${\tt starFormation[Disks|Spheroids]VelocityExponent} are input parameters, $\tau_{\rm dynamical}\equiv r/V$ is the dynamical timescale of the component and $r$ and $V$ are the characteristic radius and velocity respectively of the component. The timescale is not allowed to fall below a minimum value specified by {\tt starFormation[Disks|Spheroids]MinimumTimescale} (in Gyr).

\subsection{Kennicutt-Schmidt}\label{sec:StarFormationKennicuttSchmidt}\index{star formation!Kennicutt-Schmidt law}

Selected with {\tt starFormationTimescaleDisksMethod}$=${\tt Kennicutt-Schmidt} this method assumes that the Kennicutt-Schmidt law holds \citep{schmidt_rate_1959,kennicutt_global_1998}:
\begin{equation}
\dot{\Sigma}_\star = A \left({\Sigma_{\rm H} \over M_\odot \hbox{pc}^{-2}} \right)^N,
\end{equation}
where $A=${\tt [starFormationKennicuttSchmidtNormalization]} and $N=${\tt [starFormationKennicuttSchmidtExponent]} are parameters. Optionally, if the {\tt [starFormationKennicuttSchmidtTruncate]} parameter is set to true, then the star formation rate is truncated below a critical surface density such that
\begin{equation}
\dot{\Sigma}_\star = \left\{ \begin{array}{ll} A \left({\Sigma_{\rm H} \over M_\odot \hbox{pc}^{-2}} \right)^N & \hbox{ if } \Sigma_{\rm gas,disk} > \Sigma_{\rm crit} \\ A \left({\Sigma_{\rm H} \over M_\odot \hbox{pc}^{-2}} \right)^N \left(\Sigma_{\rm gas,disk}/\Sigma_{\rm crit}\right)^\alpha & \hbox{ otherwise.} \end{array} \right.
\end{equation}
Here, $\alpha=${\tt [starFormationKennicuttSchmidtExponentTruncated]} and $\Sigma_{\rm crit}$ is a critical surface density for star formation which we specify as
\begin{equation}
\Sigma_{\rm crit} = {q_{\rm crit} \kappa \sigma_{\rm gas} \over \pi \G},
\end{equation}
where $\kappa$ is the epicyclic frequency in the disk, $\sigma_{\rm gas}$ is the velocity dispersion of gas in the disk and $q_{\rm crit}=${\tt [toomreParameterCritical]} is a dimensionless constant of order unity which controls where the critical density occurs. We assume that $\sigma_{\rm gas}$ is a constant equal to {\tt [velocityDispersionDiskGas]} and that the disk has a flat rotation curve such that $\kappa = \sqrt{2} V/R$. The integrated star formation rate over the disk is then
\begin{equation}
\Psi = 2 \pi \int_0^\infty \dot{\Sigma}_\star(R) R \d R.
\end{equation}
From this the star formation timescale is found through the definition $\tau_\star \equiv M_{\rm gas, disk}/\Psi$.

\subsection{Blitz-Rosolowsky}\label{sec:StarFormationBlitzRosolowsky}\index{star formation!Blitz-Rosolowsky rule}

Selected with {\tt starFormationTimescaleDisksMethod}$=${\tt Blitz-Rosolowsky} this method assumes that the star formation rate is given by \citep{blitz_role_2006}:
\begin{equation}
 \dot{\Sigma}_\star(R) = \nu_{\rm SF}(R) \Sigma_{\rm H_2, disk}(R),
\end{equation}
where $\nu_{\rm SF}$ is a frequency given by
\begin{equation}
 \nu_{\rm SF}(R) = \nu_{\rm SF,0} \left[ 1 + \left({\Sigma_{\rm HI}\over \Sigma_0}\right)^q \right],
\end{equation}
where $q=${\tt [surfaceDensityExponentBlitzRosolowsky]} and $\Sigma_0=${\tt [surfaceDensityCriticalBlitzRosolowsky]} are parameters and the surface density of molecular gas $\Sigma_{\rm H_2} = (P_{\rm ext}/P_0)^\alpha \Sigma_{\rm HI}$, where $\alpha=${\tt [pressureExponentBlitzRosolowsky]} and $P_0=${\tt [pressureCharacteristicBlitzRosolowsky]} are parameters and the hydrostatic pressure in the disk plane assuming location isothermal gas and stellar components is given by
\begin{equation}
 P_{\rm ext} \approx {\pi\over 2} \G \Sigma_{\rm gas} \left[ \Sigma_{\rm gas} + \left({\sigma_{\rm gas}\over \sigma_\star}\right)\Sigma_\star\right]
\end{equation}
where we assume that the velocity dispersion in the gas is fixed at $\sigma_{\rm gas}=${\tt [velocityDispersionDiskGas]} and, assuming $\Sigma_\star \gg \Sigma_{\rm gas}$, we can write the stellar velocity dispersion in terms of the disk scale height, $h_\star$, as
\begin{equation}
 \sigma_\star = \sqrt{\pi \G h_\star \Sigma_\star}
\end{equation}
where we assume $h_\star/R_{\rm disk}=${\tt [heightToRadialScaleDisk]}. The star formation rate integated over the disk is
\begin{equation}
 \Psi = \int_0^\infty 2 \pi R \dot{\Sigma}_\star(R) \d R.
\end{equation}
From this the star formation timescale is found through the definition $\tau_\star \equiv M_{\rm gas, disk}/\Psi$.

\subsection{Krumholz-McKee-Tumlinson}\label{sec:StarFormationKMT09}\index{star formation!Krumholz-McKee-Tumlinson method}

Selected with {\tt starFormationTimescaleDisksMethod}$=${\tt KMT09} this method assumes that the star formation rate is given by \citep{krumholz_star_2009}:
\begin{equation}
 \dot{\Sigma}_\star(R) = \nu_{\rm SF} f_{\rm H_2}(R)\Sigma_{\rm HI, disk}(R) \left\{ \begin{array}{ll} (\Sigma_{\rm HI}/\Sigma_0)^{-1/3}, &  \hbox{ if } \Sigma_{\rm HI}/\Sigma_0 \le 1 \\ (\Sigma_{\rm HI}/\Sigma_0)^{1/3}, & \hbox{ if } \Sigma_{\rm HI}/\Sigma_0 > 1 \end{array} \right. ,
\end{equation}
where $\nu_{\rm SF}=${\tt [starFormationFrequencyKMT09]} is a frequency and $\Sigma_0=85 M_\odot \hbox{pc}^{-2}$. The molecular fraction is given by
\begin{equation}
 f_{\rm H_2} = 1 - \left( 1 + \left[ { 3 s \over 4 (1+\delta)} \right]^{-5} \right)^{-1/5},
\end{equation}
where
\begin{equation}
 \delta = 0.0712 \left[ 0.1 s^{-1} + 0.675 \right]^{-2.8},
\end{equation}
and
\begin{equation}
 s = {\ln(1+0.6\chi+0.01\chi^2) \over 0.04 \Sigma_{\rm comp,0} Z^\prime},
\end{equation}
with
\begin{equation}
 \chi = 0.77 \left[ 1 + 3.1 Z^{\prime 0.365} \right],
\end{equation}
and $\Sigma_{\rm comp,0}=c \Sigma_{\rm HI}/M_\odot \hbox{pc}^{-2}$ where $c=${\tt [molecularComplexClumpingFactorKMT09]} is a density enhancement factor relating the surface density of molecular complexes to the gas density on larger scales. Alternatively, if {\tt [molecularFractionFastKMT09]} is set to true, the molecular fraction will be computed using the faster (but less acccurate at low molecular fraction) formula
\begin{equation}
 f_{\rm H_2} = 1 - { 3s/4 \over (1 + s/4)}.
\end{equation}
The star formation rate integated over the disk is
\begin{equation}
 \Psi = \int_0^\infty 2 \pi R \dot{\Sigma}_\star(R) \d R.
\end{equation}
From this the star formation timescale is found through the definition $\tau_\star \equiv M_{\rm gas, disk}/\Psi$.

\subsection{Baugh et al. (2005)}

Selected with {\tt starFormationTimescaleDisksMethod}$=${\tt Baugh2005} this method assumes that the star formation rate is given by a modified version of the \cite{baugh_can_2005} prescription:
\begin{equation}
\tau_\star = \tau_0 (V_{\rm disk}/200\hbox{km/s})^\alpha a^\beta
\end{equation}
where $\tau_0=${\tt [starFormationDiskTimescale]}, $\alpha=${\tt [starFormationDiskVelocityExponent]} and $\beta=${\tt [starFormationExpansionExponent]}.

\section{Stellar Population Properties}\label{sec:StellarPopulationProperties}\index{stellar populations}

Algorithms for determining stellar population properties---essentially the rates of change of stellar and gas mass and abundances given a star formation rate and fuel abundances (and perhaps a historical record of star formation in the component)---are selected by the {\tt stellarPopulationPropertiesMethod} parameter.

\subsection{Instantaneous}

Selected with {\tt stellarPopulationPropertiesMethod}$=${\tt instantaneous} this method uses the instantaneous recycling approximation. Specifically, given a star formation rate $\phi$, this method assumes a rate of increase of stellar mass of $\dot{M}_\star=(1-R)\phi$, a corresponding rate of decrease in fuel mass. The rate of change of the metal content of stars follows from the fuel metallicity, while that of the fuel changes according to
\begin{equation}
 \dot{M}_{fuel,Z} = - (1-R) Z_{\rm fuel} \phi + p \phi.
\end{equation}
In the above $R$ is the instantaneous recycled fraction and $p$ is the yield, both of which are supplied by the \IMF\ subsystem. The rate of energy input from the stellar population is computed assuming that the canonical amount of energy from a single stellar population (as defined by the {\tt feedbackEnergyInputAtInfinityCanonical}) is input instantaneously.

\subsection{Noninstantaneous}

Selected with {\tt stellarPopulationPropertiesMethod}$=${\tt noninstantaneous} this method assumes fully non-instantaneous recycling and metal enrichment. Recycling and metal production rates from simple stellar populations are computed, for any given \IMF, from stellar evolution models. The rates of change are then:
\begin{eqnarray}
 \dot{M}_\star &=& \phi - \int_0^t \phi(t^\prime) \dot{R}(t-t^\prime;Z_{\rm fuel}[t^\prime]) \d t^\prime, \\
 \dot{M}_{\rm fuel} &=& -\phi + \int_0^t \phi(t^\prime) \dot{R}(t-t^\prime;Z_{\rm fuel}[t]) \d t^\prime, \\
 \dot{M}_{\star,Z} &=& Z_{\rm fuel} \phi - \int_0^t \phi(t^\prime) Z_{\rm fuel}(t^\prime)  \dot{R}(t-t^\prime;Z_{\rm fuel}[t^\prime]) \d t^\prime, \\
 \dot{M}_{{\rm fuel},Z} &=& -Z_{\rm fuel} \phi + \int_0^t  \phi(t^\prime) \{ Z_{\rm fuel}(t^\prime) \dot{R}(t-t^\prime;Z_{\rm fuel}[t^\prime]) + \dot{p}(t-t^\prime;Z_{\rm fuel}[t^\prime]) \} \d t^\prime, \\
\end{eqnarray}
where $\dot{R}(t;Z)$ and $\dot{p}(t;Z)$ are the recycling and metal yield rates respectively from a stellar population of age $t$ and metallicity $Z$. The energy input rate is computed self-consistently from the star formation history.

\section{Stellar Population Spectra}

Stellar population spectra are used to construct intgrated spectra of galaxies. The method used to compute such spectra is specified by the {\tt stellarPopulationSpectraMethod} parameter.

\subsection{Conroy, White \& Gunn}

Selected with {\tt stellarPopulationSpectraMethod}$=${\tt Conroy, White \& Gunn} this method uses v2.2 of the \href{http://www.cfa.harvard.edu/~cconroy/FSPS.html}{{\tt FSPS}} code of \cite{conroy_propagation_2009} to compute stellar spectra. If necessary, the {\tt FSPS} code will be downloaded, patched and compiled and run to generate spectra. These tabulations are then stored to file for later retrieval.

\subsection{File}

Selected with {\tt stellarPopulationSpectraMethod}$=${\tt file} this method reads stellar population spectra from an HDF5 file, with format described in \S\ref{sec:StellarPopulationSpectra}.

\section{Stellar Population Spectra Postprocessing}

Stellar population spectra are postprocessed (to handle, for example, absorption by the \IGM). The method used to postprocess spectra is specified by the {\tt stellarPopulationSpectraPostprocessMethod} parameter.

\subsection{Meiksin (2006) IGM Attenuation}

Selected with {\tt stellarPopulationSpectraPostprocessMethod}$=${\tt Meiksin2006} this method postprocesses spectra through absorption by the \IGM\ using the results of \cite{meiksin_colour_2006}.

\subsection{Madau (1995) IGM Attenuation}

Selected with {\tt stellarPopulationSpectraPostprocessMethod}$=${\tt Madau1995} this method postprocesses spectra through absorption by the \IGM\ using the results of \cite{madau_radiative_1995}.

\subsection{Null Method}

Selected with {\tt stellarPopulationSpectraPostprocessMethod}$=${\tt null} this method performs no postprocessing.

\section{Stellar Astrophysics}

Various properties related to stellar astrophysics are required by \glc. The following documents their implementation.

\subsection{Basics}

This subset of properties include recycled mass, metal yield and lifetime.  The method used to compute such properties is specified by the {\tt stellarAstrophysicsMethod} parameter.

\subsubsection{File}\label{sec:StellarAstrophysicsFile}

Selected with {\tt stellarAstrophysicsMethod}$=${\tt file} this method uses reads properties of individual stars of different initial mass and metallicity from an XML file and interpolates in them. The stars can be irregularly spaced in the plane of initial mass and metallicity. The XML file should have the following structure:
\begin{verbatim}
 <stars>
  <star>
    <initialMass>0.6</initialMass>
    <lifetime>28.19</lifetime>
    <metallicity>0.0000</metallicity>
    <ejectedMass>7.65</ejectedMass>
    <metalYieldMass>0.44435954</metalYieldMass>
    <elementYieldMassFe>2.2017e-13</elementYieldMassFe>
    <source>Table 2 of Tumlinson, Shull &amp; Venkatesan (2003, ApJ, 584, 608)</source>
    <url>http://adsabs.harvard.edu/abs/2003ApJ...584..608T</url>
  </star>
  <star>
    .
    .
    .
  </star>
  .
  .
  .
 </stars
\end{verbatim}
Each {\tt star} element must contain the {\tt initialMass} (given in $M_\odot$) and {\tt metallicity} tags. Other tags are optional. {\tt lifetime} gives the lifetime of such a star (in Gyr), {\tt ejectedMass} gives the total mass (in $M_\odot$) ejected by such a star during its lifetime, {\tt metalYieldMass} gives the total mass of metals yielded by the star during its lifetime while {\tt elementYieldMassX} gives the mass of element {\tt X} yielded by the star during its lifetime. The {\tt source} and {\tt url} tags are not used, but are strongly recommended to provide a reference to the origin of the stellar data.

\subsection{Stellar Winds}

Energy input to the \ISM\ from stellar winds is used in calculations of feedback efficiency. The method used to compute stellar wind properties is specified by the {\tt stellarWindsMethod} parameter.

\subsubsection{Leitherer et al. (1992)}

Selected with {\tt stellarWindsMethod}$=${\tt Leitherer1992} this method uses the fitting formulae of \cite{leitherer_deposition_1992} to compute stellar wind energy input from the luminosity and effective temperature of a star.

\subsection{Stellar Tracks}

The method used to compute stellar tracks is specified by the {\tt stellarTracksMethod} parameter.

\subsubsection{File}\label{sec:StellarTracksFile}

Selected with {\tt stellarTracksMethod}$=${\tt file} in this method luminosities and effective temperatures of stars are computed from a tabulated set of stellar tracks. The file containing the tracks to use is specified via the {\tt stellarTracksFile} parameter. The file specified must be an HDF5 file with the following structure:
\begin{verbatim}
 stellarTracksFile
  |
  +-> metallicity1
  |    |
  |    +-> metallicity
  |    |
  |    +-> mass1
  |    |    |
  |    |    +-> mass
  |    |    |
  |    |    +-> age
  |    |    |
  |    |    +-> luminosity
  |    |    |
  |    |    +-> effectiveTemperature
  |    |
  |    x-> massN
  |
  x-> metallicityN
\end{verbatim}
Each {\tt metallicityN} group tabulates tracks for a given metallicity (the value of which is stored in the {\tt metallicity} dataset within each group), and may contain an arbitrary number of {\tt massN} groups. Each {\tt massN} group should contain a track for a star of some mass (the value of which is given in the {\tt mass} dataset). Within each track three datasets specify the {\tt age} (in Gyr), {\tt luminosity} (in $L_\odot$) and {\tt effectiveTemperature} (in Kelvin) along the track.

\subsection{Supernovae Type Ia}\index{supernovae!Type Ia}

Properties of Type Ia supernovae, including the cumulative number occuring and metal yield, are handled by the method selected using the {\tt supernovaeIaMethod} parameter.

\subsubsection{Nagashima et al. (2005) Prescription}

Selected with {\tt supernovaeIaMethod}$=${\tt Nagashima} this method uses the prescriptions from \cite{nagashima_metal_2005} to compute the numbers and yields of Type Ia supernovae.

\subsection{Population III Supernovae}\index{supernovae!Population III}\index{Population III!supernovae}

Properties of Population III specific supernovae are handled by the method selected with the {\tt supernovaePopIIIMethod} parameter.

\subsubsection{Heger \& Woosley (2002)}

Selected with {\tt supernovaePopIIIMethod}$=${\tt Heger + Woosley} this method computes the energies of pair instability supernovae from the results of \cite{heger_nucleosynthetic_2002}.

\subsection{Stellar Feedback}

Aspects of stellar feedback are computed by the method selected with the {\tt stellarFeedbackMethod} parameter.

\subsubsection{Standard}

Selected with {\tt stellarFeedbackMethod}$=${\tt standard}, the method assumes that the cumulative energy input from a stellar population is equal to the total number of (Type II and Type Ia) supernovae multiplied by {\tt supernovaEnergy} (specified in ergs) plus any Population III-specific supernovae energy plus the integrated energy input from stellar winds. The minimum mass of a star required to form a Type II supernova is specified (in $M_\odot$) via the {\tt initialMassForSupernovaeTypeII} parameter.

\section{Substructure and Merging}\index{merging!substructure}\index{substructure}

Substructures and merging of nodes/substructures is controlled by several algorithms which are described below:

\subsection{Merging Timescales}\index{merging!dynamical friction}

The method used to compute merging timescales of substructures is specified by the {\tt satelliteMergingMethod} parameter.

\subsubsection{Dynamical Friction: Lacey \& Cole}

Selected with {\tt satelliteMergingMethod}$=${\tt Lacey-Cole}, this method computes merging timescales using the dynamical friction calculation of \cite{lacey_merger_1993}. Timescales are multiplied by the value of the {\tt mergingTimescaleMultiplier} input parameter.

\subsubsection{Dynamical Friction: Lacey \& Cole $+$ Tormen}

Selected with {\tt satelliteMergingMethod}$=${\tt Lacey-Cole + Tormen}, this method computes merging timescales using the dynamical friction calculation of \cite{lacey_merger_1993} with a parameterization of orbital parameters designed to fit the results of \cite{tormen_rise_1997} as described by \cite{cole_hierarchical_2000}. Timescales are multiplied by the value of the {\tt mergingTimescaleMultiplier} input parameter. Specifically, the merging time is taken to be:
\begin{equation}
 \tau_{\rm merge} = {f_\tau \Phi \tau_{\rm dynamical} \over 2 B(1)} { M_{\rm host}/M_{\rm satellite} \over \ln (M_{\rm host}/M_{\rm satellite})}
\end{equation}
where $f_\tau=${\tt mergingTimescaleMultiplier}, $\tau_{\rm dynamical}$ is the dynamical time of the host halo and $B(x)=\hbox{erf}(x)-2 x \exp(x)/\sqrt{\Pi}$. The orbital factor $\Phi \equiv \epsilon^{0.78} (R_{\rm c}/R_{\rm virial})^2$ is drawn at random from a log-normal distribution with median $-0.14$ and dispersion $0.26$ as found by \cite{cole_hierarchical_2000}.

\subsubsection{Dynamical Friction: Jiang (2008)}

Selected with {\tt satelliteMergingMethod}$=${\tt Jiang2008}, this method computes merging timescales using the dynamical friction calibration of \cite{jiang_fitting_2008}.

\subsubsection{Dynamical Friction: Boylan-Kolchin (2008)}

Selected with {\tt satelliteMergingMethod}$=${\tt BoylanKolchin2008}, this method computes merging timescales using the dynamical friction calibration of \cite{boylan-kolchin_dynamical_2008}.

\subsection{Virial Orbits}

The algorithm to be used to determine orbital parameters of substructures when they first enter the virial radius of their host is specified via the {\tt virialOrbitsMethod} parameter.

\subsubsection{Benson (2005)}

Selected with {\tt virialOrbitsMethod}$=${\tt Benson2005}, this method selects orbital parameters randomly from the distribution given by \cite{benson_orbital_2005}.

\subsubsection{Wetzel (2010)}

Selected with {\tt virialOrbitsMethod}$=${\tt Wetzel2010}, this method selects orbital parameters randomly from the distribution given by \cite{wetzel_orbits_2010}, including the redshift and mass dependence of the distributions. Note that the parameter $R_1$ can be come negative (which is unphysical) for certain regimes of mass and redshift according to the fitting function for $R_1$ given by \cite{wetzel_orbits_2010}. Therefore, we enforce $R_1>0.05$. Similarly, the parameter $C_1$ can become very large in some regimes which is probably an artifact of the fitting function used rather than physically meaningful (and which causes numerical difficulties in evaluating the distribution). We therefore prevent $C_1$ from exceeding $9.999999$\footnote{We use this value rather than $10$ since the GSL $_2F_1$ hypergeomtric function fails in some cases when $C_1\ge 10$.}

\subsection{Node Merging}

The algorithm to be used to process nodes when they become substructures is specified by the {\tt nodeMergersMethod} parameter.

\subsubsection{Single Level Hierarchy}

Selcted with {\tt nodeMergersMethod}$=${\tt single level hierarchy}, this method maintains a single level hierarchy of substructure, i.e. it tracks only substructures, not sub-substructures or deeper levels. When a node first becomes a satellite it is appended to the list of satellites associated with its host halo. If the node contains its own satellites they will be detached from the node and appended to the list of satellites of the new host (and assigned new merging times).

\section{Supernovae Feedback Models}\label{sec:sneFeedback}\index{supernovae!feedback}\index{feedback}

The supernovae feedback driven outflow rate is computed using the method specified by the {\tt starFormationFeedback[Disks|Spheroids]Method} for disks and spheroids respectively.

\subsection{Power Law}

Selected with {\tt starFormationFeedback[Disks|Spheroids]Method}$=${\tt power law}, this method assumes an outflow rate of:
\begin{equation}
 \dot{M}_{\rm outflow} = \left({V_{\rm outflow} \over V}\right)^{\alpha_{\rm outflow}} {\dot{E} \over E_{\rm canonical}},
\end{equation}
where $V_{\rm outflow}=${\tt [disk|spheroid]OutflowVelocity} (in km/s) and $\alpha_{\rm outflow}=${\tt [disk|spheroid]OutflowVelocity} are input parameters, $V$ is the characteristic velocity of the component, $\dot{E}$ is the rate of energy input from stellar populations and $E_{\rm canonical}$ is the total energy input by a canonical stellar population normalized to $1 M_\odot$ after infinite time.

\section{Supernovae Expulsive Feedback Models}\label{sec:sneExpulsiveFeedback}\index{supernovae!feedback}\index{feedback!expulsive}

The expulsive supernovae feedback driven outflow rate is computed using the method specified by the {\tt starFormationExpulsiveFeedback[Disks|Spheroids]Method} for disks and spheroids respectively.

\subsection{Null}

Selected with {\tt starFormationExpulsiveFeedback[Disks|Spheroids]Method}$=${\tt null}, this method assumes a zero outflow rate.

\subsection{Superwind}

Selected with {\tt starFormationExpulsiveFeedback[Disks|Spheroids]Method}$=${\tt superwind}, this method assumes an outflow rate of:
\begin{equation}
 \dot{M}_{\rm outflow} =  \beta_{\rm superwind} {\dot{E} \over E_{\rm canonical}} \left\{ \begin{array}{ll} \left( V_{\rm superwind}/V\right)^2 & \hbox{ if } V > V_{\rm superwind} \\ 1 & \hbox{ otherwise,} \end{array} \right.
\end{equation}
where $V_{\rm superwind}=${\tt [disk|spheroid]SuperwindVelocity} (in km/s) and $\beta_{\rm superwind}=${\tt [disk|spheroid]SuperwindMassLoading} are input parameters, $V$ is the characteristic velocity of the component, $\dot{E}$ is the rate of energy input from stellar populations and $E_{\rm canonical}$ is the total energy input by a canonical stellar population normalized to $1 M_\odot$ after infinite time.

\section{Supermassive Black Holes Binary Mergers}\index{supermassive black holes!mergers}

The method to be used for computing the effects of binary mergers of supermassive black holes is specified by the {\tt blackHoleBinaryMergersMethod} parameter.

\subsection{Rezzolla et al. (2008)}

Selected with {\tt blackHoleBinaryMergersMethod}$=${\tt Rezzolla2008}, this method uses the fitting function of \cite{rezzolla_final_2008} to compute the spin of the black hole resulting from a binary merger. The mass of the resulting black hole is assumed to equal the sum of the mass of the initial black holes (i.e. there is negligible energy loss through gravitational waves).

