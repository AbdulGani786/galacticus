\chapter{Extracting and Analyzing Results}

\glc\ stores its output in an \href{http://www.hdfgroup.org/HDF5/}{HDF5} file. The contents of this file can be viewed and manipulated using a variety of ways including:
\begin{description}
 \item[\href{http://www.hdfgroup.org/hdf-java-html/hdfview/}{{\normalfont \scshape HDFView}}] This is a graphical viewer for exploring the contents of HDF5 files;
 \item[\href{https://portal.hdfgroup.org/display/HDF5/HDF5+Command-line+Tools}{HDF5 Command Line Tools}] A set of tools which can be used to extract data from HDF5 files (\href{https://portal.hdfgroup.org/display/HDF5/h5dump}{{\normalfont \ttfamily h5dump}} and \href{https://portal.hdfgroup.org/display/HDF5/h5ls}{{\normalfont \ttfamily h5ls}} are particularly useful);
 \item[\href{https://portal.hdfgroup.org/pages/viewpage.action?pageId=50073943}{C Fortran 90 APIs}] Allow access to, and manipulation of data in HDF5 files;
 \item[\href{https://www.h5py.org/}{{\normalfont \scshape h5py}}] A Python interface to HDF5 files.
\end{description}

In the remainder of this section the structure of \glc\ HDF5 files is described.

\section{General Structure of Output File}

Figure~\ref{fig:glcOutputFileStructure} shows the structure of a typical \glc\ output file. The various groups and subgroups are described below.

\begin{figure}
\begin{center}
\begin{verbatim}
outputFile.hdf5
 |
 +-> UUID                                     Attribute {1}
 |
 +-> Build                                    Group
 |    |
 |    +-> FoX_library_version                 Attribute {1}
 |    +-> GSL_library_version                 Attribute {1}
 |    +-> HDF5_library_version                Attribute {1}
 |    +-> make_CCOMPILER                      Attribute {1}
 |    +-> make_CCOMPILER_VERSION              Attribute {1}
 |    +-> make_CFLAGS                         Attribute {1}
 |    +-> make_CPPCOMPILER                    Attribute {1}
 |    +-> make_CPPCOMPILER_VERSION            Attribute {1}
 |    +-> make_CPPFLAGS                       Attribute {1}
 |    +-> make_FCCOMPILER                     Attribute {1}
 |    +-> make_FCCOMPILER_VERSION             Attribute {1}
 |    +-> make_FCFLAGS                        Attribute {1}
 |    +-> make_FCFLAGS_NOOPT                  Attribute {1}
 |    +-> make_MODULETYPE                     Attribute {1}
 |    +-> make_PREPROCESSOR                   Attribute {1}
 |    +-> sourceChangeSetDiff                 Dataset   {1}
 |    +-> sourceChangeSetMerge                Dataset   {1}
 |
 +-> Outputs                                  Group
 |    |
 |    +-> Output1                             Group
 |    |    |
 |    |    +-> nodeData                       Group
 |    |    |     |
 |    |    |     +-> nodeProperty1            Dataset {<nodeCount>}
 |    |    |     +-> ...                      Dataset {<nodeCount>}
 |    |    |     +-> ...                      Dataset {<nodeCount>}
 |    |    |     +-> ...                      Dataset {<nodeCount>}
 |    |    |     +-> nodePropertyN            Dataset {<nodeCount>}
 |    |    |
 |    |    +-> mergerTreeCount                Dataset {<treeCount>}
 |    |    |
 |    |    +-> mergerTreeIndex                Dataset {<treeCount>}
 |    |    |
 |    |    +-> mergerTreeStartIndex           Dataset {<treeCount>}
 |    |    |
 |    |    +-> mergerTreeWeight               Dataset {<treeCount>}
 |    |    |
 |    |    +-> mergerTree1                    Group              [optional]
 |    |    |     |
 |    |    |     +-> nodeProperty1            Reference
 |    |    |     +-> ...                      Reference
 |    |    |     +-> ...                      Reference
 |    |    |     +-> ...                      Reference
 |    |    |     +-> nodePropertyN            Reference
 |    |    |
 |    |    x-> ...                            Group              [optional]
 |    |    x-> ...                            Group              [optional]
 |    |    x-> ...                            Group              [optional]
 |    |    x-> mergerTree<treeCount>          Group              [optional]
 |    |    |
 |    |    +-> outputExpansionFactor          Attribute {1}
 |    |    +-> outputTime                     Attribute {1}
 |    |
 |    x-> Output2                             Group
 |
 +-> Filters                                  Group
 |    |
 |    +-> name                                Dataset   {<filterCount>}
 |    +-> wavelengthEffective                 Dataset   {<filterCount>}
 |
 +-> Parameters                               Group
 |    |
 |    +-> inputParameter1                     Attribute {}
 |    +-> ...                                 Attribute {}
 |    +-> ...                                 Attribute {}
 |    +-> ...                                 Attribute {}
 |    +-> inputParameterN                     Attribute {}
 |    +-> inputParameter1                     Group
 |         |
 |         +-> subInputParameter1             Attribute {}
 |         +-> ...                            Attribute {}
 |         +-> subInputParameterN             Attribute {}
 |    x-> ...                                 Attribute {}
 |    x-> ...                                 Attribute {}
 |    x-> ...                                 Attribute {}
 |    x-> inputParameterN                     Group
 |
 +-> Version                                  Group
 |    |
 |    +-> runTime                             Attribute {1}
 |    +-> versionMajor                        Attribute {1}
 |    +-> versionMinor                        Attribute {1}
 |    +-> versionRevision                     Attribute {1}
 |    +-> hgRevision                          Attribute {1}
 |    +-> hgHash                              Attribute {1}
 |    +-> runByName                           Attribute {1}
 |    +-> runByEmail                          Attribute {1}
 |
 +-> globalHistory                            Group
      |
      +-> historyExpansion                    Dataset {<historyCount>}
      +-> historyStarFormationRate            Dataset {<historyCount>}
      +-> historyTime                         Dataset {<historyCount>}
\end{verbatim}
\end{center}
\caption{Structure of a \glc\ HDF5 output file. {\normalfont \ttfamily <treeCount>} is the total number of merger trees present in a given output, and {\normalfont \ttfamily <nodeCount} is the total number of nodes (in all trees) present in an output.}
\label{fig:glcOutputFileStructure}
\end{figure}

\subsection{UUID}\label{sec:UUID}

The UUID (\href{https://secure.wikimedia.org/wikipedia/en/wiki/Universally_unique_identifier}{Universally Unique Identifier}) is a unique identifier assigned to each \glc\ model that is run. It allows identification of a given model and can be referenced from, for example, an external database.

\subsection{Build Information}\label{sec:BuildInformation}

\glc\ automatically stores various information about how it was built in the {\normalfont \ttfamily Build} group attributes. Currently, included attributes consist of:
\begin{description}
\item[{\normalfont \ttfamily FoX\_library\_version}] The version number of the FoX library;
\item[{\normalfont \ttfamily GSL\_library\_version}] The version number of the GSL library;
\item[{\normalfont \ttfamily HDF5\_library\_version}] The version number of the HDF5 library;
\item[{\normalfont \ttfamily make\_CCOMPILER}] The C compiler command used;
\item[{\normalfont \ttfamily make\_CCOMPILER\_VERSION}] The C compiler version information;
\item[{\normalfont \ttfamily make\_CFLAGS}] The flags passed to the C compiler;
\item[{\normalfont \ttfamily make\_CPPCOMPILER}] The C++ compiler command used;
\item[{\normalfont \ttfamily make\_CPPCOMPILER\_VERSION}] The C++ compiler version information;
\item[{\normalfont \ttfamily make\_CPPFLAGS}] The flags passed to the C++ compiler;
\item[{\normalfont \ttfamily make\_FCCOMPILER}] The Fortran compiler command used;
\item[{\normalfont \ttfamily make\_FCCOMPILER\_VERSION}] The Fortran compiler version information;
\item[{\normalfont \ttfamily make\_FCFLAGS}] The flags passed to the Fortran compiler;
\item[{\normalfont \ttfamily make\_FCFLAGS\_NOOPT}] The flags passed to the Fortran compiler for unoptimized compiles;
\item[{\normalfont \ttfamily make\_MODULETYPE}] The Fortran module type identifier string;
\item[{\normalfont \ttfamily make\_PREPROCESSOR}] The preprocessor command used.
\end{description}

Additionally, two datasets are included which store details of the \glc\ source changeset. {\normalfont \ttfamily sourceChangeSetMerge} contains the output of ``{\normalfont \ttfamily hg bundle create HEAD \^origin}'', that is, it contains a Git archive that incorporates any changes made to the current branch relative to the main \glc\ branch. {\normalfont \ttfamily sourceChangeSetDiff} contains the output of ``{\normalfont \ttfamily git diff}'', that is, all differences between the source code in the working directory and that which has been committed to Git. Used together, these two datasets allow the precise source code used to run the model to be recovered from the main branch \glc\ source.

\subsection{Filters}

For each broadband filter used in the \glc\ model run an entry is added to the datasets in this group. Currently, two datasets are generated:
\begin{description}
\item[{\normalfont \ttfamily name}] The name of each filter used.
\item[{\normalfont \ttfamily wavelengthEffective}] The effective wavelength, $\lambda_\mathrm{eff}$ (defined as $\lambda_\mathrm{eff}=\left. \int_0^\infty \lambda R(\lambda) \mathrm{d}\lambda \right/ \int_0^\infty R(\lambda) \mathrm{d}\lambda$, where $R(\lambda)$ is the filter response) of the filter in \AA.
\end{description}

\subsection{Parameters}

The {\normalfont \ttfamily Parameters} group contains a record of all parameter values (either input or default) that were used for this \glc\ run. The group contains a long list of attributes, each attribute named for the corresponding parameter and with a single entry giving the value of that parameter. If a parameter has subparameters, a group is created having the same name as the parameter, which will contain attributes corresponding to each subparameter.

\subsection{Version}

The {\normalfont \ttfamily Version} group contains a record of the \glc\ version used for this model, storing the major and minor version numbers, the revision number and the {\normalfont \scshape Git} branch and hash (if the code is being maintained using {\normalfont \scshape Git}, otherwise a value of ``{\normalfont \ttfamily unknown}'' is entered). Additionally, the time at which the model was run is stored and, if the {\normalfont \ttfamily galacticusConfig.xml} file (see \S\ref{sec:ConfigFile}) is present and contains contact details, the name and e-mail address of the person who ran the model.

\subsection{globalHistory}\label{sec:globalHistory}\hyperdef{sec}{globalHistory}{}\index{history!global}\index{outputs!global history}

The {\normalfont \ttfamily globalHistory} group stores volume averaged properties of the model universe as a function of time. Currently, the properties stored are:
\begin{description}
 \item[{\normalfont \ttfamily historyTime}] Cosmic time (in Gyr);
 \item[{\normalfont \ttfamily historyExpansion}] Expansion factor;
 \item[{\normalfont \ttfamily historyStarFormationRate}] Volume averaged star formation rate (in $M_\odot/$Gyr/Mpc$^3$).
 \item[{\normalfont \ttfamily historyDiskStarFormationRate}] Volume averaged star formation rate in disks (in $M_\odot/$Gyr/Mpc$^3$).
 \item[{\normalfont \ttfamily historySpheroidStarFormationRate}] Volume averaged star formation rate in spheroids (in $M_\odot/$Gyr/Mpc$^3$).
 \item[{\normalfont \ttfamily historyStellarDensity}] Volume averaged stellar mass density (in $M_\odot/$Mpc$^3$).
 \item[{\normalfont \ttfamily historyDiskStellarDensity}] Volume averaged stellar mass density in disks (in $M_\odot/$Mpc$^3$).
 \item[{\normalfont \ttfamily historySpheroidStellarDensity}] Volume averaged stellar mass density in spheroids (in $M_\odot/$Mpc$^3$).
 \item[{\normalfont \ttfamily historyGasDensity}] Volume averaged cooled gas density (in $M_\odot/$Mpc$^3$).
 \item[{\normalfont \ttfamily historyNodeDensity}] Volume averaged resolved node density (in $M_\odot/$Mpc$^3$).
\end{description}
Dimensionful datasets have a {\normalfont \ttfamily unitsInSI} attribute which gives their units\index{units} in the SI system.

\subsection{Outputs}

The {\normalfont \ttfamily Outputs} group contains one or more sub-groups corresponding to the output times requested from \glc. Each sub-group contains the following information:
\begin{description}
 \item[{\normalfont \ttfamily outputTime} \emph{(attribute)}] The cosmic time (in Gyr) at this output;
 \item[{\normalfont \ttfamily outputExpansionFactor} \emph{(attribute)}] The expansion factor at this output;
 \item[{\normalfont \ttfamily nodeData}] A group of node properties as described below.
 \item[{\normalfont \ttfamily mergerTree} subgroups \emph{(optional)}] A set of {\normalfont \ttfamily mergerTree} groups as described below.
\end{description}

Output is controlled by parameters given within the {\normalfont \ttfamily mergerTreeOutput} section of the parameter file. Current options are:
\begin{description}
\item[{\normalfont \ttfamily outputMergerTrees}] If {\normalfont \ttfamily true} then each merger tree is output to the relevant sub-group at each output time (see \S\ref{sec:nodeDataGroup}). Otherwise merger trees are not output. [Default: {\normalfont \ttfamily true}.]
\item[{\normalfont \ttfamily outputReferences}] If {\normalfont \ttfamily true} then an HDF5 reference dataset is written for each merger tree subgroup (see \S\ref{sec:mergerTreeSubgroups}). [Default: {\normalfont \ttfamily false}.]
\item[{\normalfont \ttfamily galacticFilterMethod}] A \href{https://github.com/galacticusorg/galacticus/releases/download/masterRelease/Galacticus_Development.pdf#methods.galacticFilter}{\normalfont \ttfamily galacticFilter} which is applied to each node in the tree to determine whether or not it should be output. By combining multiple filters it is possible to construct arbitrarily complex criteria for output. [Default: {\normalfont \ttfamily always}.]
\end{description}

\subsubsection{nodeData group}\label{sec:nodeDataGroup}\hyperdef{sec}{nodeDataGroup}{}

The {\normalfont \ttfamily nodeData} group contains all data from nodes in all merger trees. The group consists of a collection of datasets each of which lists a property of all nodes in the trees which exist at the output time. Where relevant, each dataset contains an attribute, {\normalfont \ttfamily unitsInSI}, which gives the units\index{units} of the dataset in the SI system.

\subsubsection{mergerTree datasets}\label{sec:mergerTreeDatasets}

To allow locating of nodes belonging to a given merger tree in the datasets in the {\normalfont \ttfamily nodeData} group, the {\normalfont \ttfamily mergerTreeStartIndex} and {\normalfont \ttfamily mergerTreeCount} datasets list the starting index of each tree's nodes in the {\normalfont \ttfamily nodeData} datasets, and the number of nodes belonging to each tree respectively. Additionally, the {\normalfont \ttfamily mergerTreeWeight} dataset lists the {\normalfont \ttfamily volumeWeight} property for each tree (see \S\ref{sec:mergerTreeSubgroups}) which gives the weight (in Mpc$^{-3}$) which should be assigned to this tree (and all nodes in it) to create a volume-averaged sample (see \S\ref{sec:volumeLimitedSamples}). Finally, the {\normalfont \ttfamily mergerTreeIndex} dataset gives the index of each tree stored in the {\normalfont \ttfamily nodeData} datasets.

\subsubsection{mergerTree subgroups}\label{sec:mergerTreeSubgroups}

These subgroups will be present if the {\normalfont \ttfamily [mergerTreeOutputReferences]} parameter is set to true. Each {\normalfont \ttfamily mergerTree} subgroup contains HDF5 references to all data on a single merger tree. The group consists of a collection of scalar references each of which points to the appropriate region of the corresponding dataset in the {\normalfont \ttfamily nodeData} group. Additionally, the {\normalfont \ttfamily volumeWeight} attribute of this group gives the weight (in Mpc$^{-3}$) which should be assigned to this tree (and all nodes in it) to create a volume-averaged sample. (A second attribute, {\normalfont \ttfamily volumeWeightUnitsInSI}, gives the units of {\normalfont \ttfamily volumeWeight} in the SI system.)

\section{Topics in Analysis of \glc\ Outputs}

\subsection{Building Volume Limited Samples}\label{sec:volumeLimitedSamples}\index{samples!volume limited}\index{galaxies!weighting}\index{{\normalfont \ttfamily mergerTreeWeight}@mergerTreeWeight}

The {\normalfont \ttfamily mergerTreeWeight} property (see \S\ref{sec:mergerTreeDatasets}) property specifies the weight to be assigned to each merger tree in a model to construct a representative (i.e. volume limited) sample of galaxies. \glc\ does not typically generate every merger tree in a fixed volume of the Universe (as an N-body simulation might for example) as it's generally a waste of time to simulate millions of low mass halos and only a small number of high mass halos. The {\normalfont \ttfamily mergerTreeWeight} factors correct for this sampling. If merger trees are being built, then the {\normalfont \ttfamily mergerTreeWeight}, $w_i$, for each tree of mass $M_i$ (where the trees are ranked in order of increasing mass) is given by
\begin{equation}
 w_i = \int_{M_\mathrm{min}}^{M_\mathrm{max}} n(M) \mathrm{d}M,
\end{equation}
where $n(M)$ is the dark matter halo mass function and
\begin{eqnarray}
 M_\mathrm{min} &=& \sqrt{M_{i-1}M_i}, \\
 M_\mathrm{min} &=& \sqrt{M_i M_{i+1}}.
\end{eqnarray}

Suppose, for example, that we wish to construct a luminosity function of galaxies. In particular, we consider a luminosity bin $k$ which extends from $L_k-\Delta k/2$ to $L_k+\Delta k/2$. If tree $i$ contains $N_i$ galaxies with luminosities $l_{i,j}$, where $j$ runs from $1$ to $N_i$, then the luminosity function in this bin is given by:
\begin{equation}
 \phi_k = \sum_i \sum_{j=1}^{N_i} \left\{ \begin{array}{ll} w_i & \hbox{ if  } L_k-\Delta k/2 < l_{i,j} \le L_k+\Delta k/2 \\ 0 & \hbox{ otherwise.} \end{array} \right.
\end{equation}
