\chapter{Definitions and Conventions Used in \glc}

\glc\ adopts various definitions and conventions internally. These are explained below.

\section{Luminosity Units}

Galaxy luminosities are output in the \gls{ABmagnitude} system, such that a luminosity of $1$ corresponds to an object of $0^{\mathrm th}$ absolute magnitude in the \gls{ABmagnitude} system. This implies that the luminosities are in units of $4.4659\times 10^{13}$~W/Hz.

\section{Peculiar Velocities}\label{sec:GalacticusVelocityDefinitions}

Velocities in \glc\ are always \emph{physical} velocities. When reading merger tree properties (including velocities) from file it is often convenient to store velocities without the Hubble flow contribution, as ``peculiar velocities'', in the file---see \S\ref{sec:ForestHalosGroup} for how to specify whether or not  the velocities included in the file include the Hubble flow or not.

If peculiar velocities are stored it is important to use the same definition of pecular velocity as is used by \glc. Defininf $t$ to be physical time and ${\mathbf x}$ to be comoving position, \glc\ uses the conventional definition of peculiar velocity in a cosmological context, namely that it is the deviation of the physical velocity from the Hubble flow. Physical coordinates are given by ${\mathbf r} = a{\mathbf x}$, so the peculiar velocity is
\begin{equation}
{\mathbf v}_{\mathrm pec} \equiv {{\mathrm d} {\mathbf r} \over {\mathrm d} t} - H {\mathbf r} = a {{\mathrm d} {\mathbf x}\over{\mathrm d} t} = {{\mathrm d}{\mathbf x}\over{\mathrm d}\eta},
\end{equation}
where ${\mathrm d}\eta = {\mathrm d}t/a$ is conformal time. 

\section{Gravitational Potentials}\index{potential!gravitational}\index{gravitational potential}

Gravitational potentials are measured in velocity units (i.e. km$^2$/s$^2$), and the arbitrary constant offset is chosen such that the total gravitational potential in any halo at the virial radius is $\Phi(r_{\rm virial})=-V_{\rm virial}^2$. This choice is made for two reasons:
\begin{enumerate}
\item some mass distributions used have potentials which diverge as $r\rightarrow\infty$, so the usual choice of $\Phi(r) \rightarrow 0$ as $r \rightarrow \infty$ is not applicable;
\item this choice is consistent with the potential at the virial radius of the halo considered as a point mass as is used in Keplerian orbit calculations.
\end{enumerate}
Note that the choice constant offset for the potential of any mass distribution or galactic component is irrelevant---the galactic structure function which computes potential will ensure that the potential is always offset to match the definition given above.
