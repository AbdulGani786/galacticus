\chapter{Constraining {\normalfont \scshape Galacticus}}

\section{Constraint File}

\glc\ has a complete constraints infrastructure which implements various \gls{mcmc} algorithms to analyze the posterior probability distribution of the model given some compilation of constraints. The infrastructure is \gls{mpi} parallelized and ideal for running on large compute clusters.

To perform a constraint calculation simply build the constraint code:
\begin{verbatim}
 make Constrain_Galacticus.exe
\end{verbatim}
and run with a parameter file and configuration file. Typically, you will want to run this code under \gls{mpi}, for example:
\begin{verbatim}
 mpirun -n 4 Constrain_Galacticus.exe mcmcParameters.xml mcmcConfig.xml
\end{verbatim}
would run 4 processes (typically you will need to run many more than this). If running on a \gls{pbs} queue, embed this command in a suitable \gls{pbs} script and submit. 

The parameter file follows the same format as a standard \glc\ parameter file and specifies the values of parameters to be used. For example, the seed used fo pseudo-random number sequences can be specified in this file. 

The configuration file specifies the details of the constraint simulation to be performed. An example configuration file is:
\begin{verbatim}
<?xml version="1.0" encoding="UTF-8"?>
<simulationConfig>

  <likelihood>
    <type>Galacticus</type>
    <name>verySimplisticToStellarMassFunction</name>
    <compilation>stellarMassFunction_SDSS_z0.07.xml</compilation>
    <baseParameters>./mcmcWork/verySimplisticToStellarMassFunctionBase.xml</baseParameters>
    <workDirectory>./mcmcWork</workDirectory>
    <scratchDirectory>./mcmcScratch</scratchDirectory>
    <report>no</report>
    <randomize>no</randomize>
    <threads>4</threads>
    <saveState>no</saveState>
    <cpulimit>1200</cpulimit>
    <coredump>NO</coredump>
    <coredumpsize>0</coredumpsize>
    <sequentialModels>no</sequentialModels>
    <memoryLimit>2gb</memoryLimit>
    <environment>LD_LIBRARY_PATH=/opt/gcc-trunk/lib:/opt/gcc-trunk/lib64:/usr/local/upstream/lib:$LD_LIBRARY_PATH</environment>
    <environment>PATH=/opt/gcc-trunk/bin:$PATH</environment>
  </likelihood>

  <convergence>
    <type>GelmanRubin</type>
    <Rhat>1.2</Rhat>
    <burnCount>100</burnCount>
    <testCount>100</testCount>
    <outlierCountMaximum>0</outlierCountMaximum>
    <outlierSignificance>0.95</outlierSignificance>
    <outlierLogLikelihoodOffset>60</outlierLogLikelihoodOffset>
  </convergence>
  
  <state>
    <type>history</type>
    <acceptedStateCount>100</acceptedStateCount>
  </state>
  
  <proposalSize>
    <type>adaptive</type>
    <gammaInitial>1.77</gammaInitial>
    <gammaFactor>1.414</gammaFactor>
    <acceptanceRateMinimum>0.4</acceptanceRateMinimum>
    <acceptanceRateMaximum>0.6</acceptanceRateMaximum>
    <updateCount>10</updateCount>
  </proposalSize>
  
  <randomJump>
    <type>adaptive</type>
  </randomJump>
  
  <simulation>
    <type>temperedDifferentialEvolution</type>
    <stepsMaximum>1000000</stepsMaximum>
    <stepsPostConvergence>100000</stepsPostConvergence>
    <acceptanceAverageCount>100</acceptanceAverageCount>
    <logFileRoot>./mcmcWork/mcmc/chains</logFileRoot>
    <temperatureMaximum>64.0</temperatureMaximum>
    <untemperedStepCount>20</untemperedStepCount>
    <temperedLevels>10</temperedLevels>
    <stepsPerLevel>10</stepsPerLevel>
    <logFlushCount>10<logFlushCount>
  </simulation>

  <parameters>
    <parameter>
      <name>starFormationTimescaleDisksHaloScalingVirialVelocityExponent</name>
      <prior>
	<distribution>
	  <type>uniform</type>
	  <minimum>-6.0</minimum>
	  <maximum>+0.0</maximum>
	</distribution>
      </prior>
      <mapping>
        <type>linear</type>
      </mapping>
      <random>
	<type>Cauchy</type>
	<median>0.0</median>
	<scale>0.006</scale>
      </random>
    </parameter>
    <parameter>
      <name>starFormationTimescaleDisksHaloScalingRedshiftExponent</name>
      <prior>
	<distribution>
	  <type>uniform</type>
	  <minimum>-1.0</minimum>
	  <maximum>+4.0</maximum>
	</distribution> 
      </prior>
      <mapping>
        <type>linear</type>
      </mapping>
      <random>
	<type>Cauchy</type>
	<median>0.0</median>
	<scale>0.005</scale>
      </random>
    </parameter>
  </parameters>
  
</simulationConfig>
\end{verbatim}

\subsection{Parameters and Priors}\label{sec:ParametersPriors}

The {\normalfont \ttfamily parameters} section contains a list of all parameters to be varied in the analysis. Each parameter is described by one {\normalfont \ttfamily parameter} element. That element must contain a {\normalfont \ttfamily name} element, which gives the name of the parameter, a {\normalfont \ttfamily prior} element that contains a {\normalfont \ttfamily distribution} element defining the distribution for this prior, a {\normalfont \ttfamily mapping} element that describes the mapping of the parameter into the internal state used in \gls{mcmc} calculations, and (for differential evolution simulations) a {\normalfont \ttfamily random} element that defines the distribution to be used for the random perturbation to be added to this parameter in proposals.

The {\normalfont \ttfamily name} element can specify subparameters, elements in an array of parameters, and elements within a parameter's {\normalfont \ttfamily value} element. For example, consider the parameter file:
\begin{verbatim}
<parameter1 value="123"/>
<parameterA value="456"/>
<parameterA value="789"/>
<parameterA value="987"/>
<parameter2 value="abc">
 <parameter2a value="def"/>
</parameter2>
<parameterX value="1.0 2.0 3.0"/>
\end{verbatim}
\begin{itemize}
\item A name of {\normalfont \ttfamily parameter1} will change the value ({\normalfont \ttfamily 123}) of the {\normalfont \ttfamily parameter1} element;
\item A name of {\normalfont \ttfamily parameterA[1]} will change the value ({\normalfont \ttfamily 1789}) of the second {\normalfont \ttfamily parameterA} element (array indexing is 0-offset);
\item A name of {\normalfont \ttfamily parameter2->parameter2a} will change the value ({\normalfont \ttfamily def}) of the {\normalfont \ttfamily parameter2a} element;
\item A name of {\normalfont \ttfamily parameterX{1}} will change the value ({\normalfont \ttfamily 2.0}) of second entry in the value of the {\normalfont \ttfamily parameterX} element.
\end{itemize}

Currently allowed mappings are:
\begin{itemize}
\item[{\normalfont \ttfamily linear}] Effectively a null mapping, as the parameter is mapped into itself: $x \rightarrow x$.
\item[{\normalfont \ttfamily logarithmic}] The parameter is mapped logarithmically: $x \rightarrow \log(x)$. This mapping can be applied only to uniform priors.
\end{itemize}
Note that the mapping will map the values of the prior. For example, if you specify a uniform prior with a logarithmic mapping, the upper and lower limits of the prior should be specified on $x$, not on $\log(x)$. These limits will be mapped appropriately internally.

\subsubsection{Loading External Parameters/Priors}

It is also possible to load parameters and their priors from external files. This is useful to add common sets of parameters, such as cosmological parameter. To do so, add an element of the form:
\begin{verbatim}
<xi:include href="../../constraints/parameters/wmap9Cosmology.xml" 
   xmlns:xi="http://www.w3.org/2001/XInclude" />
\end{verbatim}
\emph{after} the {\normalfont \ttfamily parameters} section of the constraint file. The {\normalfont \ttfamily href} attribute must give the path (relative to the constraint file, or absolute) to the external parameter file. This file should contain its own {\normalfont \ttfamily parameters} block, describing all parameters to be varied along with their priors. 

\subsubsection{Derived Parameter Values}

It is possible to define parameters in terms of other parameters. Common uses for this include:
\begin{itemize}
 \item Setting $\Omega_\Lambda$ from the value of $\Omega_\mathrm{M}$ to enforce a flat Universe;
 \item Setting the values of parameters with correlated priors as linear combinations of dummy parameters for which the priors are independent.
\end{itemize}
To define a parameter in this way include a {\normalfont \ttfamily parameter} element of the form:
\begin{verbatim}
<parameter>
 <name>sigma_8</name>
 <define>0.8178+%cosmology0*0.003817+%cosmology1*0.007931+%cosmology2*0.01002
    +%cosmology3*0.001584+%cosmology4*0.002931+%cosmology5*0.001727</define>
</parameter>
\end{verbatim}
Here the {\normalfont \ttfamily define} element gives an equation for the parameter in terms of other parameters. All standard mathematical operators and functions (as recognized by Perl) can be used, and other parameters referenced by using their name prefixed with a ``\%''.

\subsubsection{Including External Parameters}

Predefined sets of parameters (along with their priors) can be included using the {\normalfont \ttfamily xi:include} element. For example,
\begin{verbatim}
 <xi:include href="../../constraints/parameters/wmap7Cosmology.xml"
    xmlns:xi="http://www.w3.org/2001/XInclude" />
\end{verbatim}
will include a set of parameters from the file {\normalfont \ttfamily ../../constraints/parameters/wmap7Cosmology.xml} which defines priors on cosmological parameters consistent with the covariance matrix of the WMAP-7 cosmological constraints \citep{komatsu_seven-year_2010}.
