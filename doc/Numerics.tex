\chapter{Numerical Implementation}

\section{Timestepping Criteria}\label{sec:TimesteppingCriteria}\index{timesteps!criteria}

\glc\ evolves each merger tree forest by repeatedly walking the trees and evolving each node forward in time by some timestep $\Delta t$. Nodes are evolved individually such that nodes in different branches of a tree may have reached different cosmic times at any given point in the execution of \glc. Each node is evolved over the interval $\Delta t$ using an adaptive \gls{ode} solver, which adjusts the smaller timesteps, $\delta t$, taken in evolving the system of \glspl{ode} to maintain a specified precision.

The choice of $\Delta t$ then depends on other considerations. For example, a node should not be evolved beyond the time at which it is due to merge with another galaxy. Also, we typically don't want satellite nodes to evolve too far ahead of their host node, such that any interactions between satellite and host occur (near) synchronously.

In the remainder of this section we list all criteria used to select $\Delta t$ for a node. All criteria are considered and the largest $\Delta t$ consistent with all criteria is selected.

\subsection{Tree Criteria}

The following timestep criteria ensure that tree evolution occurs in a way which correctly preserves tree structure and ordering of interactions between \glspl{node}.

\subsubsection{``Branch Segment'' Criteria}

For \glspl{node} which are the \gls{primary progenitor} of their \gls{parent}, the ``branch segment'' criterion asserts that
\begin{equation}
 \Delta t \le t_\mathrm{parent} - t
\end{equation}
where $t$ is current time in the \gls{node} and $t_\mathrm{parent}$ is the time of the \gls{parent} \gls{node}. This ensures that \gls{primary progenitor} \glspl{node} to not evolve beyond the time at which their \gls{parent} (which they will replace) exists.  If this criterion is the limiting criteria for $\Delta t$ then the \gls{node} will be promoted to replace its \gls{parent} at the end of the timestep. 

\subsubsection{``Parent'' Criteria}

For \glspl{node} which are satellites in a hosting \gls{node} the ``\gls{parent}'' timestep criterion asserts that
\begin{eqnarray}
\Delta t &\le& t_\mathrm{host}, \\
\Delta t &\le& \epsilon_\mathrm{host} (a/\dot{a}),
\end{eqnarray}
where $t_\mathrm{host}=${\normalfont \ttfamily [timestepHostAbsolute]}, $\epsilon_\mathrm{host}=${\normalfont \ttfamily [timestepHostRelative]}, and $a$ is expansion factor. These criteria are intended to prevent a satellite for evolving too far ahead of the host node before the host is allowed to ``catch up''.

\subsubsection{``Satellite'' Criteria}

For \glspl{node} which host satellite \glspl{node}, the ``satellite'' criterion asserts that
\begin{equation}
 \Delta t \le \hbox{min}(t_\mathrm{satellite}) - t,
\end{equation}
where $t$ is the time of the host \gls{node} and $t_\mathrm{satellite}$ are the times of all satellite \glspl{node} in the host. This criterion prevents a host from evolving ahead of any satellites.

\subsubsection{``Sibling'' Criteria}

For \glspl{node} which are \glspl{primary progenitor}, the ``sibling'' criterion asserts that
\begin{equation}
 \Delta t \le \hbox{min}(t_\mathrm{sibling}) - t,
\end{equation}
where $t$ is the time of the host \gls{node} and $t_\mathrm{sibling}$ are the times of all siblings of the \gls{node}. This criterion prevents a \gls{node} from reaching its \gls{parent} (and being promoted to replace it) before all of its siblings have reach the \gls{parent} and have become satellites within it.

\subsubsection{``Mergee'' Criteria}

For \glspl{node} with \gls{mergee} \glspl{node}, the ``\gls{mergee}'' criterion asserts that
\begin{equation}
 \Delta t \le \hbox{min}(t_\mathrm{merge}) - t,
\end{equation}
where $t$ is the time of the host \gls{node} and $t_\mathrm{merge}$ are the times at which the \glspl{mergee} will merge. This criterion prevents a \gls{node} from evolving past the time at which a merger event takes place.

\subsection{General Criteria}

\subsubsection{``Simple'' Criteria}\label{phys:mergerTreeEvolveTimestep:mergerTreeEvolveTimestepSimple}

The {\normalfont \ttfamily simple} timestep criteria assert that
\begin{eqnarray}
\Delta t &\le& t_\mathrm{simple}, \\
\Delta t &\le& \epsilon_\mathrm{simple} (a/\dot{a}),
\end{eqnarray}
where $t_\mathrm{simple}=${\normalfont \ttfamily [timestepSimpleAbsolute]}, $\epsilon_\mathrm{simple}=${\normalfont \ttfamily [timestepSimpleRelative]}, and $a$ is expansion factor. These criteria are intended to prevent any one node evolving over an excessively large time in one step. In general, these criteria are not necessary, as nodes should be free to evolve as far as possible unless prevented by some physical requirement. These criteria are therefore present to provide a simple example of how timestep criteria work.

\subsubsection{``Satellite'' Criteria}\label{phys:mergerTreeEvolveTimestep:mergerTreeEvolveTimestepSatellite}

The {\normalfont \ttfamily satellite} timestep criteria asserts the following for satellite \glspl{node}. If the satellite's merge target has been advanced to at least a time of $t_\mathrm{required} = t_\mathrm{satellite} + \Delta t_\mathrm{merge} - \delta t_\mathrm{merge,maximum}$ then 
\begin{equation}
\Delta t \le \Delta t_\mathrm{merge},
\end{equation}
where $t_\mathrm{satellite}$ is the current time for the satellite \gls{node}, $\Delta t_\mathrm{merge}$ is the time until the satellite is due to merge and $\delta t_\mathrm{merge,maximum}$ is the maximum allowed time difference between merging galaxies. This ensures that the satellite is not evolved past the time at which it is due to merge. If this criterion is the limiting criteria for $\Delta t$ then the merging of the satellite will be triggered at the end of the timestep.

If the merge target has not been advanced to at least $t_\mathrm{required}$ then instead
\begin{equation}
\Delta t \le \hbox{max}(\Delta t_\mathrm{merge}-\delta t_\mathrm{merge,maximum}/2,0),
\end{equation}
is asserted to ensure that the satellite does not reach the time of merging until its merge target is sufficiently close (within $\delta t_\mathrm{merge,maximum}$) of the time of merging.

\subsection{Output Criteria}

\subsubsection{``History'' Criteria}\label{phys:mergerTreeEvolveTimestep:mergerTreeEvolveTimestepHistory}

The {\normalfont \ttfamily history} timestep criterion asserts that
\begin{equation}
 \Delta t \le t_{\mathrm{history},i} - t
\end{equation}
where $t$ is the current time, $t_{\mathrm{history},i}$ is the $i^\mathrm{th}$ time at which the global history (see \S\ref{sec:globalHistory}) of galaxies is to be output and $i$ is chosen to be the smallest $i$ such that $t_{\mathrm{history},i} > t$. If there is no $i$ for which $t_{\mathrm{history},i} > t$ this criterion is not applied. If this criterion is the limiting criterion for $\Delta t$ then the properties of the galaxy will be accumulated to the global history arrays at the end of the timestep.

\subsubsection{``Record Main Branch Evolution'' Criteria}\label{phys:mergerTreeEvolveTimestep:mergerTreeEvolveTimestepRecordEvolution}

If selected, the {\normalfont \ttfamily recordEvolution} timestep criterion asserts that
\begin{equation}
 \Delta t \le t_{\mathrm{record},i} - t
\end{equation}
where $t$ is the current time, $t_{\mathrm{record},i}$ is the $i^\mathrm{th}$ time at which the evolution of main branch galaxies is to be output and $i$ is chosen to be the smallest $i$ such that $t_{\mathrm{record},i} > t$. If there is no $i$ for which $t_{\mathrm{record},i} > t$ this criterion is not applied. If this criterion is the limiting criterion for $\Delta t$ then the properties of the galaxy will be recorded at the end of the timestep.
