% Acronyms.
\newacronym{cdm}{CDM}{cold dark matter}
\newacronym{cmb}{CMB}{cosmic microwave background}
\newacronym{igm}{IGM}{intergalactic medium}
\newacronym{imf}{IMF}{initial mass function}
\newacronym{isco}{ISCO}{innermost stable circular orbit}
\newacronym{ism}{ISM}{interstellar medium}
\newacronym{ode}{ODE}{ordinary differential equation}
\newacronym{nfw}{NFW}{Navarro-Frenk-White (dark matter halo profile)}
\newacronym{sed}{SED}{spectral energy distribution}
\newacronym{sne}{SNe}{supernovae}
\newglossaryentry{adaf}{type=\acronymtype, name={ADAF}, description=\glslink{adafg}{advection-dominated accretion flow}, first={advection-dominated accretion flow (ADAF)}, see=[Glossary:]{adafg}}

% Glossary entries.
\newglossaryentry{component}{name={component}, description={An individual physical system within a \gls{node}, such as a dark matter halo, a galactic disk or a supermassive black hole}}

\newglossaryentry{forest}{name={forest}, description={A collection of merger trees that are linked together by virtue of nodes which jump between trees}}

\newglossaryentry{node}{name={node}, description={A single point in a merger tree, consisting of a dark matter halo and associated baryons}}

\newglossaryentry{mergee}{name={mergee}, description={For a given node in a merger tree, the set of mergee nodes consists of all nodes which will undergo a galaxy merger with the node at some point in the future}}

\newglossaryentry{primary progenitor}{name={primary progenitor}, description={The progenitor of a given \gls{node} which is regarding as the direct descendent of that \gls{node} (often, but not always, the most massive progenitor). Other progenitors are considered to merge into this primary progenitor}}

\newglossaryentry{parent}{name={parent}, description={In a merger tree, the parent node of any given node that exists at time $t_0$ is that node to which it is directly connected in the tree at time $t_1>t_0$}}

\newglossaryentry{Bernoulli distribution}{name={Bernoulli distribution}, description={A discrete probability distribution which takes value $1$ with success probability $p$ and value $0$ with failure probability $q = 1-p$. Read more on \href{http://en.wikipedia.org/wiki/Bernoulli_distribution}{Wikipedia}}}

\newglossaryentry{UUID}{name={UUID}, description={A \href{http://en.wikipedia.org/wiki/Universally_unique_identifier}{universally unique identifier}---this is a label which uniquely identifies some object (in this case, a \glc\ model)}}

\newglossaryentry{ABmagnitude}{name={AB magnitude}, description={An astronomical magnitude system in which the apparent magnitude is defined as $m=-2.5\log_{10}f-48.60$ for a flux density, $f$, measured in ergs per second per square centimeter per hertz}}

\newglossaryentry{forwardDescendent}{name={forward descendent}, description={The node with which the mass (or majority of the mass) of a node will become associated with at a later time. This type of descendent is usually relevant when considering how halos and galaxies evolve forward in time and should be distinguished from a \gls{backwardDescendent} which is relevant when building merger trees}}

\newglossaryentry{backwardDescendent}{name={backward descendent}, description={The \gls{primary progenitor} of a node. This type of descendent is usually relevant when building merger trees and should be distinguished from a \gls{forwardDescendent} which is relevant when considering how halos and galaxies evolve forward in time}}

\newglossaryentry{MD5hash}{name={MD5 hash}, description={The \href{http://en.wikipedia.org/wiki/MD5}{MD5 Message-Digest Algorithm} is a widely used cryptographic hash function that produces a 128-bit (16-byte) hash value. In \glc\ it is used to encode unique labels for modules which are incorporated into file names. \glc\ uses the \href{http://www.gnu.org/software/libc/}{\tt glibc} \href{http://en.wikipedia.org/wiki/Crypt_(Unix)}{\tt crypt()} function to compute MD5 hashes, but swicthes ``{\tt /}'' for ``{\tt @}'' in the hash (since ``{\tt /}'' is inconvenient for use in file names)}}

\newglossaryentry{adafg}{name={ADAF},
    description={An advection-dominated accretion flow (ADAF) is a particular solution for an accretion flow around a black hole, star or compact object in which energy liberated by viscous forces is stored within the accretion flow and advected inward to the central object (see \citealt{narayan_advection-dominated_1998})}}
