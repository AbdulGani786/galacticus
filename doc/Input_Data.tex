\chapter{Input Data}

In some configurations, \glc\ requires additional input data to run. For example, if asked to process galaxy formation through a set of externally derived merger trees, then a file describing those trees must be given. In the remainder of this section we describe the structure of external datasets which can be inputs to \glc.

\section{Merger Tree Files}\index{merger trees!files}

\glc\ can read merger tree structures from file, and then form and evolve galaxies within those files. The most common use case involving such files is when merger trees are extracted from an N-body simulation. \glc\ can read several formats of merger tree data file, but its preferred format is that described \href{https://github.com/galacticusorg/galacticus/wiki/Merger-Tree-File-Format}{here}. A tool ``{\normalfont \ttfamily rockstar2galacticus}\footnote{Developed primarily by Markus Haider.}'' is available to convert merger trees from the \href{https://bitbucket.org/gfcstanford/rockstar}{\normalfont \ttfamily Rockstar}/\href{https://bitbucket.org/pbehroozi/consistent-trees}{\normalfont \ttfamily ConsistentTrees} tree builders into \glc's preferred format.

\section{Broadband Filters}\index{filters!broadband}

To compute luminosities through a given filter, \glc\ requires the response function, $R(\lambda)$, of that filter to be defined. \glc\ follows the convention of \cite{hogg_k_2002} in defining the filter response to be the fraction of incident photons received by the detector at a given wavelength, multiplied by the relative photon response (which will be 1 for a photon-counting detector such as a CCD, or proportional to the photon energy for a bolometer/calorimeter type detector. Filter response files are stored in the {\normalfont \ttfamily static/filters/} subdirectory of the \glc\ datasets. A large number of filters are provided in that location, but it is straightforward to add new filters. The structure of a filter definition file is shown below, with the {\normalfont \ttfamily SDSS\_g.xml} filter response file used as an example:
\begin{verbatim}
 <filter>
  <description>SDSS g vacuum (filter+CCD +0 air mass)</description>
  <name>SDSS g</name>
  <origin>Michael Blanton</origin>
  <response>
    <datum>   3630.000      0.0000000E+00</datum>
    <datum>   3680.000      2.2690000E-03</datum>
    <datum>   3730.000      5.4120002E-03</datum>
    <datum>   3780.000      9.8719997E-03</datum>
    <datum>   3830.000      2.9449999E-02</datum>
    .
    .
    . 
  </response>
  <effectiveWavelength>4727.02994472695</effectiveWavelength>
  <vegaOffset>0.107430167298754</vegaOffset>
</filter>
\end{verbatim}
The {\normalfont \ttfamily description} element should provide a description of the filter, while the {\normalfont \ttfamily name} element provides a shorter name. The {\normalfont \ttfamily origin} element should describe from where/whom this filter originated. The {\normalfont \ttfamily response} element contains a list of {\normalfont \ttfamily datum} elements each giving a wavelength (in Angstroms) and response pair. The normalization of the response is arbitrary. The {\normalfont \ttfamily effectiveWavelength} element gives the mean, response-weighted wavelength of the filter and is used, for example, in dust attenuation calculations. The {\normalfont \ttfamily vegaOffset} element gives the value (in magnitudes) which must be added to an AB-system magnitude in this system to place it into the Vega system. Both {\normalfont \ttfamily effectiveWavelength} and {\normalfont \ttfamily vegaOffset} can be computed by running
\begin{verbatim}
 scripts/filters/vega_offset_effective_lambda.pl data/filters
\end{verbatim}
which will compute these values for any filter files that do not already contain them and append them to the files.
